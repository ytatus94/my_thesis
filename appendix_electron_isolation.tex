The electron reconstruction, identification, and isolation play a crucial role for many ATLAS analysis.
The electrons\footnote{The electrons and positron are referred to as electrons.} leave tracks in the ID and energy deposits in the ECAL.
The reconstruction algorithm combines the signals in the calorimeter and the tracks in the ID to defined the electron candidates.
The reconstructed candidates are identified as electrons based on a likelihood discrimination which distingishs the electron candidates from the hadrons, non-prompt electrons originating from photon conversions, and heavy flavour hadron decays.
Additionally, the electron candidates are required to be isolated to further distinguish the signal and the background objects.
The electron efficiency measurements are performed based on the tag-and-probe method using the $Z \to ee$ and $J/\psi \to ee$ samples.

%%%
%%%
%%%

\section{Electron reconstruction}
\label{sec:app_electron_reconstruction}
The electrons are reconstructed in the central region of the ATLAS detector ($|\eta| < 2.47$).
%%%
%%%
%%%

\section{Electron identification}
\label{sec:app_electron_identification}

%%%
%%%
%%%

\section{Electron isolation}
\label{sec:app_electron_isolation}
