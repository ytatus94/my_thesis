The European Organization for Nuclear Research (CERN\footnote{The name CERN is derived from the acronym for the French Conseil Europ\'{e}en pour la Recherch Nucl\'{e}aire}) was founded in 1954 and is based in the suburb of Geneva on the Franco\textendash Swiss border.
The main function of CERN is to provide particle accelerators and detectors for high-energy physics research.
The physicists and engineers at CERN are probing the fundamental structure of the universe using the world's largest and most complex scientific facility \textemdash \ the Large Hadron Collider (LHC)~\cite{1748-0221-3-08-S08001}.
In the LHC, the particles are boosted to high energies and collide at close to the speed of light.
The results of the collisions are recorded by the various detectors.
There are seven experiments at the LHC.
The biggest of these experiments are ATLAS (A Toroidal LHC ApparatuS)~\cite{1748-0221-3-08-S08003} and CMS (Compact Muon Solenoid)~\cite{1748-0221-3-08-S08004} which use general-purpose detectors to investigate a broad physics programme ranging from the search for the Higgs boson to extra dimensions and particles that could make up dark matter.
The ALICE (A Large Ion Collider Experiment)~\cite{1748-0221-3-08-S08002} experiment is designed to study the physics of quark-gluon plasma form and the LHCb (Large Hadron Collider beauty)~\cite{1748-0221-3-08-S08005} experiment specializes in investigating of CP violation by studying the $b$-quark.
These four detectors sit underground in huge caverns of the LHC ring.
The rest three experiments, TOTEM~\cite{1748-0221-3-08-S08007}, LHCf~\cite{1748-0221-3-08-S08006}, and MoEDAL~\cite{Pinfold:1181486}, are smaller.
The TOTEM (TOTal Elastic and diffractive cross section Measurement)~\cite{1748-0221-3-08-S08007} experiment aims at the measurement of total cross section, elastic scattering, and diffractive dissociation.
The LHCf (Large Hadron Collider forward)~\cite{1748-0221-3-08-S08006} experiment is intended to measure the neutral particle produced by the collider using the forward particles.
The prime motivation of the MoEDAL (Monopole and Exotics Detector at the LHC)~\cite{Pinfold:1181486} experiment is to search directly for the magnetic monopole.

\section{The Large Hadron Collide}
The LHC~\cite{1748-0221-3-08-S08001} is the world's largest and most powerful accelerator which accelerates and collides protons in a 26.7 km circumference crossing the Franco\textendash Swiss border 100 m underground.
The LHC is designed for collisions at a centre-of-mass energy $\sqrt{s}=14$ TeV and an instantaneous luminosity of $\mathcal{L} =10^{34} \ \textrm{cm}^{-2}\textrm{s}^{-1}$.
The first beam was circulated through the collider on the morning of 10 September 2008~\cite{CERN-COURIER-Sep192008}.
However, a magnet quench incident occurred on 19 September 2008 and caused extensive damage to over 50 superconducting magnets, their mountings, and the vacuum pipe.
Most of 2009 was spent on repairs the damage caused by the magnet quench incident and the operations resumed on 20 November of that year.
The first phase of data-taking (Run 1) started at the end of 2009 and the beam energy was increased to a centre-of-mass $\sqrt{s}=7$ TeV in 2011 and $\sqrt{s} = 8$ TeV in 2012.
The total integrated luminosity of 5.46 fb$^{-1}$ was collected in 2011 and of 22.8 fb$^{-1}$ was collected in 2012.
Since 13 February 2013 the LHC was in the Long Shutdown 1 (LS1) phase for maintenance and upgrades.
On 5 April 2015, the LHC restarted and was operating at a centre-of-mass energy $\sqrt{s}=13$ TeV throughout the Run 2 phase (2015 to 2017).



\section{The ATLAS experiment}

\subsection{The Inner Detector}

\subsection{The Calorimeter}

\subsection{The Muon Spectrometer}

\subsection{The Trigger system}

\subsection{}