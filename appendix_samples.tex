The Monte Carlo (MC) simulated samples are used to model the SUSY signals and to estimate the SM background.
The MC samples are processed using a ATLAS detector full simulation or a fast simulation based on {\GEANT}4~\cite{} simulation package.
The full simulation simulates the detailed properties of the ATLAS detector while the fast simulation uses a parameterized calorimeter response and simulates ID and MS based on {\GEANT}4.

%%%
%%%
%%%

\section{Samples used for strong interaction}
\label{sec:app_samples_strong}

Table~\ref{tab:app_sample_strong} shows the event generator, parton shower, cross-section normalization, PDF set, and the set of tunned parameters for modelling for all samples.

\begin{table}[htp]
%\begin{center}
\resizebox{\textwidth}{!}{% <------ Don't forget this %
\begin{tabular}{cccccc}
\hline
\hline
Physics process & Event generator & Parton shower & Cross-section normalization & PDF set & Set of tunned parameters\\
\hline
\hline
$t\bar{t}+X$\\
$t\bar{t}W, t\bar{t}Z/\gamma^{*}$ & MG5\_{\scriptsize A}MC@NLO 2.2.2 & {\PYTHIA} 8.186 & NLO & NNPDF2.3LO & A14\\
$t\bar{t}H$ & MG5\_{\scriptsize A}MC@NLO 2.3.2 & {\PYTHIA} 8.186 & NLO & NNPDF2.3LO & A14\\
4$t$ & MG5\_{\scriptsize A}MC@NLO 2.2.2 & {\PYTHIA} 8.186 & NLO & NNPDF2.3LO & A14\\
Dibosno\\
$ZZ, WZ$ & {\SHERPA} 2.2.1 & {\SHERPA} 2.2.1 & NLO & NNPDF2.3LO & {\SHERPA} default\\
Other (inc. $W^{\pm}W^{\pm}$) & {\SHERPA} 2.1.1 & {\SHERPA} 2.1.1 & NLO & CT10 & {\SHERPA} default\\
Rare\\
$t\bar{t}WW, t\bar{t}WZ$ & MG5\_{\scriptsize A}MC@NLO 2.2.2 & {\PYTHIA} 8.186 & NLO & NNPDF2.3LO & A14\\
$tZ, tWZ, tt\bar{t}$  & MG5\_{\scriptsize A}MC@NLO 2.2.2 & {\PYTHIA} 8.186 & LO & NNPDF2.3LO & A14\\
$WH, ZH$ & MG5\_{\scriptsize A}MC@NLO 2.2.2 & {\PYTHIA} 8.186 & NLO & NNPDF2.3LO & A14\\
Triboson & {\SHERPA} 2.1.1 & {\SHERPA} 2.1.1 & NLO & CT10 & {\SHERPA} default\\
\hline
\hline
\end{tabular}
%\end{center}
}
\caption{}
\label{tab:app_sample_strong}
\end{table}%


%%%
%%%
%%%

\section{Samples used for weak interaction}
\label{sec:app_samples}

%%%
%%%
%%%