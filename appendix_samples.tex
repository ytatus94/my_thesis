The Monte Carlo (MC) samples are used to model the SUSY signals and to estimate the SM background.
The MC samples are processed using a ATLAS detector full simulation (FullSim) or a fast simulation (AFII\footnote{AFII stands for ATLAS Fast II.}) based on {\GEANT}4~\cite{Agostinelli:2002hh} simulation package.
The FullSim simulates the detailed properties of the ATLAS detector while the AFII uses a parameterized calorimeter response and simulates ID and MS~\cite{ATLAS:1300517} based on {\GEANT}4.
The simulated MC events are reweighted to the observed pile-up conditions in the data.

%%%
%%%
%%%

\section{Samples used for strong interaction}
\label{sec:app_samples_strong}

Table~\ref{tab:app_sample_strong} shows the event generator, parton shower, cross-section normalization, PDF set~\cite{Martin:2009iq}, and the set of tunned parameters for modelling for all samples.
Except those produced by the {\SHERPA}, the \textsc{EvtGEN}\xspace v1.2.0 package~\cite{Lange:2001uf} is used to model the properties of bottom and charm hadron decays for all MC samples.

\begin{table}[htp]
%\begin{center}
\resizebox{\textwidth}{!}{% <------ Don't forget this %
\begin{tabular}{ccccccc}
\hline
\hline
Signal/Background                                      & Physics process                   & Event generator                  & Parton shower     & Cross-section normalization & PDF set    & Set of tunned parameters\\
\hline
\hline
\multirow{3}{*}{Signal}                                & RPC                               & MG5\_{\scriptsize A}MC@NLO 2.2.3 & {\PYTHIA} 8.186   & NLO+NLL                     & NNPDF2.3LO & A14\\
                                                       & RPV (except Figure~\ref{})        & MG5\_{\scriptsize A}MC@NLO 2.2.3 & {\PYTHIA} 8.210   & or                          & NNPDF2.3LO & A14\\
                                                       & RPV (Figure~\ref{})               & {\HERWIGpp} 2.7.1                & {\HERWIGpp} 2.7.1 & NLO-Prospino2               & CTEQ6L1    & UEEE5\\
\hline
\multirow{3}{*}{\shortstack{$t\bar{t}+X$\\background}} & $t\bar{t}W, t\bar{t}Z/\gamma^{*}$ & MG5\_{\scriptsize A}MC@NLO 2.2.2 & {\PYTHIA} 8.186   & NLO                         & NNPDF2.3LO & A14\\
                                                       & $t\bar{t}H$                       & MG5\_{\scriptsize A}MC@NLO 2.3.2 & {\PYTHIA} 8.186   & NLO                         & NNPDF2.3LO & A14\\
                                                       & 4$t$                              & MG5\_{\scriptsize A}MC@NLO 2.2.2 & {\PYTHIA} 8.186   & NLO                         & NNPDF2.3LO & A14\\
\hline
\multirow{2}{*}{\shortstack{Dibosno\\background}}      & $ZZ, WZ$                          & {\SHERPA} 2.2.1                  & {\SHERPA} 2.2.1   & NLO                         & NNPDF2.3LO & {\SHERPA} default\\
                                                       & Other (inc. $W^{\pm}W^{\pm}$)     & {\SHERPA} 2.1.1                  & {\SHERPA} 2.1.1   & NLO                         & CT10       & {\SHERPA} default\\
\hline
\multirow{4}{*}{\shortstack{Rare\\background}}         & $t\bar{t}WW, t\bar{t}WZ$          & MG5\_{\scriptsize A}MC@NLO 2.2.2 & {\PYTHIA} 8.186   & NLO                         & NNPDF2.3LO & A14\\
                                                       & $tZ, tWZ, tt\bar{t}$              & MG5\_{\scriptsize A}MC@NLO 2.2.2 & {\PYTHIA} 8.186   & LO                          & NNPDF2.3LO & A14\\
                                                       & $WH, ZH$                          & MG5\_{\scriptsize A}MC@NLO 2.2.2 & {\PYTHIA} 8.186   & NLO                         & NNPDF2.3LO & A14\\
                                                       & Triboson                          & {\SHERPA} 2.1.1                  & {\SHERPA} 2.1.1   & NLO                         & CT10       & {\SHERPA} default\\
\hline
\hline
\end{tabular}
%\end{center}
}
\caption{The simulated signal and background MC samples.
The event generator, parton shower, cross-section normalization, PDF set, and the set of tunned parameters for each samples are shown.
The $t\bar{t}WW, t\bar{t}WZ, tZ, tWZ, tt\bar{t}, WH, ZH$ and triboson background samples are labeled in the "rare" because they contribute a very small amount to the signal region.}
\label{tab:app_sample_strong}
\end{table}%


%%%
%%%
%%%

\section{Samples used for weak interaction}
\label{sec:app_samples}

%%%
%%%
%%%