%-------------------------------------------------------------------------------
%% outhesis_template.tex
%% version 1.0
%% Chris McRaven <mcraven@physics.ou.edu>
%%
%% 'outhesis.cls' is a class file for a master or phd thesis that
%% conforms to the requirements of the graduate college at the
%% University of Oklahoma. This file is a hacked version of 'book.cls,'
%% that includes all of the formatting requirements set forth in
%% the document 'DissertationInstPacket.pdf' available at
%% http://gradweb.ou.edu/Current/Forms/doctoral/DissertationPagination.pdf
%%
%% This class file relies on a few packages to work.  You must have the
%% following packages installed:
%%  amsfonts, amsmath, amssymb, tikz, lineno, microtype, hyperref
%%
%% Most of these packages are included in distributions of latex.  If you
%% get a lot of errors when compiling, check that these packages are
%% installed.
%%
%% By default, the class file will conform to the requirements, but three
%% options are provided for assistance in proofing the document.
%%
%% linenumbers -- turns on linenumbers in the left margin
%% summarypage -- places a page at the beginning of the document listing 
%%                the number of tables, figures, and bibliography items.
%% hyperlinks -- hyperlinks the citations and references for easier
%%                navigation in the document in a reader which supports
%%                hyperlinks.
%%
%-------------------------------------------------------------------------------

\newcommand*{\ATLASLATEXPATH}{atlas_latex/}

%\documentclass[linenumbers,summarypage,hyperlinks]{outhesis}
%\documentclass[linenumbers,hyperlinks]{outhesis}
\documentclass[hyperlinks]{outhesis}
%\documentclass{outhesis}



%-------------------------------------------------------------------------------
% Extra packages:
%-------------------------------------------------------------------------------
% See doc/atlas-physics.pdf for a list of the defined symbols.
% Default options are:
%   true:  journal, misc, particle, unit, xref
%   false: BSM, hion, math, process, other, texmf
% See the package for details on the options.

%\usepackage{\ATLASLATEXPATH atlasbiblatex} % Style file with biblatex options for ATLAS documents.
\usepackage{\ATLASLATEXPATH atlasphysics} % Useful macros
\usepackage[margin=1.0in]{geometry} % see geometry.pdf on how to lay out the page. There's lots.
%\usepackage{geometry} % see geometry.pdf on how to lay out the page. There's lots.
\usepackage{graphicx}
\usepackage{subfigure}
%\usepackage{subcaption} %These packages gives the author the ability to have subfigures within figures, or subtables within table floats.
\usepackage[labelsep=colon,labelfont=bf,font=singlespacing,width=\textwidth,compatibility=false]{caption}
\usepackage{hyperref} % for url
\usepackage{slashed} % Dirac slash notation
\usepackage{amsmath}
\usepackage{authblk}
\usepackage{multirow}
\usepackage{booktabs} % enable the use of \toprule, \midrule, and \bottomrule
\usepackage{color} % for setting font color


\geometry{a4paper} % or letter or a5paper or ... etc
% \geometry{landscape} % rotated page geometry

%\usepackage[english]{babel}
%\usepackage[utf8x]{inputenc}
%\usepackage{listings}
%\usepackage{textpos}  % for putting logo
%\usepackage{tikz}     % transparent background image
%\usepackage{eurosym}  % euro sign
%\usepackage{appendixnumberbeamer} % count presentation slides only
%\usepackage{pbox} % new line in cell of table
%\usepackage{tabularx} % can change the table width

%\newcommand{\HRule}{\rule{\linewidth}{0.3mm}}

%%%%%
%%%%%
%\usepackage[explicit]{titlesec}
%\usepackage{fix-cm}
%\usepackage{lipsum}
%% numbered
%\titleformat{\chapter}
%  {\normalfont\LARGE\bfseries\filleft}
%  {}
%  {0em}
%  {%
%    \parbox[b]{\dimexpr\linewidth-2.5cm\relax}{#1}\hfill%
%    \parbox[b]{2cm}{\hfill{\fontsize{80}{96}\selectfont\thechapter}}%
%  }
%% unnumbered
%\titleformat{name=\chapter,numberless}
%  {\normalfont\LARGE\bfseries\filleft}
%  {}
%  {0em}
%  {\parbox[b]{\dimexpr\linewidth-2.5cm\relax}{#1}}
%% spacing
%\titlespacing*{\chapter}
%  {0pt}{50pt}{40pt}
%%%%%
%%%%%  


%\usepackage[toc]{appendix}
%\usepackage{xspace}
%\usepackage{multirow,bigdelim}
%\usepackage[compat=1.1.0]{tikz-feynman}

% For a bibliography style, you must have the appropriate .bst file
% \bibliographystyle{apj}
%\bibliographystyle{prsty}
%\bibliographystyle{atlasBibStyleWithTitle}

%\usepackage{thesis-def}


%-------------------------------------------------------------------------------
% Content
%-------------------------------------------------------------------------------
 
%%% BEGIN DOCUMENT
\begin{document}

%% Place Dissertation information here
\author{Yu-Ting Shen}
\university{University of Oklahoma}
\college{Graduate College}
\department{Homer L. Dodge Department of Physics and Astronomy}
\title{This is the thesis title}
\address{Norman, Oklahoma}
\yr{2017}
\dgname{Doctor of Philosophy}
%% List your committee members here
\committee{{Dr. Patrick Skubic, Chair}, Dr. Michael Strauss, {Dr. Ron Kantowski}, Dr. Deborah Watson, {Dr. S. Lakshmivarahan}}

%% Put your dedication here. This is completely optional. Delete it if you don't need it.
\begin{dedication}
To my past, present, and future family.

\begin{quotation}
\raggedright{\emph{``You're braver than you believe, and stronger than you seem, and smarter than you think.''}} \\
\raggedleft{- A.A. Milne, Christopher Robin} 
\end{quotation}

\end{dedication}

%% Put your acknowledgements here. This is completely optional. Delete it if you don't need it.
\begin{acknowledgements}
I would like to deeply thank all of my teachers and professors over the 20 years of my academic journey, for instilling in me a deep, unquenchable desire to learn. 
\begin{quotation}
\raggedright{\emph{``If you don't know, the thing to do is not to get scared, but to learn.''}} \\
\raggedleft{- Ayn Rand, Atlas Shrugged}
\end{quotation}

I would also like to thank all of my family and friends who have helped motivate and encourage me in my pursuit of this degree. 
\begin{quotation}
\raggedright{\emph{``It's the job that's never started as takes longest to finish.''}} \\
\raggedleft{- J.R.R. Tolkien, The Lord of the Rings}
\end{quotation}

Lastly, I would like to thank my wife for not only supporting me through the entire process, but actually pushing me to be the best me I could be. Thank you for living this adventure with me!
\begin{quotation}
\raggedright{\emph{``I am a wife-made man.''}} \\
\raggedleft{- Danny Kaye} 
\end{quotation}
\begin{quotation}
\raggedright{\emph{``I knew when I met you an adventure was going to happen.''}} \\
\raggedleft{- A.A. Milne, Winnie the Pooh} 
\end{quotation}
\end{acknowledgements}

%% Put your abstract here.
\begin{abstract}
Here is the abstract
\end{abstract}

\frontmatter

\maketitle

\mainmatter

\hypersetup{linkcolor=blue}

%% You can put any part of the text in separate file with 
%% the \input{} command. This keeps the master document simpler.

\chapter{Introduction}
\label{chapter:introduction}
\graphicspath{{figures/introduction/}}
The Standard Model of particle physics (SM) describes various phenomena of particle physics.
The discovery of the Higgs boson ($H$) by the ATLAS and CMS collaboration at CERN completes the missing part of the SM prediction~\cite{Aad:2012tfa, Chatrchyan:2012xdj}.
However, there are several open challenges that cannot be explained by the SM, such as hierarchy problem~\cite{Weinberg:1975gm, Gildener:1976ai, Susskind:1978ms} and the dark matter candidate.
In order to answer those questions, a new theory extending the SM is necessary.
Supersymmetry (SUSY)~\cite{Wess:1973kz, Wess:1974tw, Golfand:1971iw, Martin:1997ns} is the most promising extensions of the SM.
SUSY, which is a spacetime symmetry, introduces the superpartners of SM particles (sparticles) with spin differing by one-half unit with respect to the SM partners.
The sparticles provide a potential solution to the hierarchy problem.
If $R$-parity is conserved~\cite{Fayet:1976et, Fayet:1977yc, Farrar:1978xj}, the sparticles are produced in pairs and the lightest SUSY particle (LSP) is stable providing a candidate for dark matter.

The charginos $\widetilde{\chi}^{\pm}_{1,2}$ and neutralinos $\widetilde{\chi}^{0}_{1,2,3,4}$ are the mass eigenstates in the order of increasing masses and collectively referred to as electroweakinos.
They are the mixture of the bino $\widetilde{B}$, winos $\widetilde{W}$, and Higgsinos $\widetilde{H}_{u,d}$ which are the superpartners of the $U(1)$, $SU(2)$ gauge bosons, and the Higgs bosons, respectively.
The charginos and neutralinos can decay into leptons and LSPs via $W$, $Z$, $H$ or sleptons $\widetilde{\ell}$.
In many SUSY models, the lightest neutralino $\widetilde{\chi}^{0}_{1}$ is the LSP.
The LSP would not be detected and results in significant missing transverse energy \MET.

The compressed scenarios refer to the small mass differences between heavier SUSY particles and the LSP.
For example, the mass differences between the heavier electroweakino states $\widetilde{\chi}^{0}_{2}$, $\widetilde{\chi}^{\pm}_{1}$ and the wino- or Higgsino-dominated LSP $\widetilde{\chi}^{0}_{1}$ range from a few {\MeV} to tens of {\GeV} depending on the composition of the mixture.
The $\widetilde{B}$, $\widetilde{W}$, and $\widetilde{H}$ composition of the $\widetilde{\chi}^{0}_{1}$ have an influence on the degree of compression.
Figure~\ref{fig:intro_LSP_composition} shows the composition of the lightest neutralino in a MSSM scan of the electroweakino sector~\cite{Aaboud:2016wna}.
Based on naturalness arguments~\cite{Barbieri:1987fn, deCarlos:1993rbr}, the Higgsino mass parameter $\mu$, the bino and wino mass parameters $M_{1}$ and $M_{2}$ satisfy $|\mu| \ll |M_{1}|, |M_{2}|$ leading to the three electroweakinos $\widetilde{\chi}^{0}_{1}$, $\widetilde{\chi}^{\pm}_{1}$, and $\widetilde{\chi}^{0}_{2}$ being dominated by the Higgsino.

\begin{figure}[htbp]
    \begin{center}
        \includegraphics[scale=1.0]{LSP_composition.pdf}
        \caption{The scatter plot in the $m(\widetilde{\chi}^{0}_{1})$ vs $m(\widetilde{\chi}^{\pm}_{1})$ plane~\cite{Aaboud:2016wna}.
        The color encodes the $\widetilde{\chi}^{0}_{1}$ composition.
        The Higgsino-dominated LSPs are colored in yellow and along the $\widetilde{\chi}^{0}_{1}$-$\widetilde{\chi}^{\pm}_{1}$ diagonal.}
        \label{fig:intro_LSP_composition}
    \end{center}
\end{figure}

This dissertation focuses on searching for electroweak production of SUSY particles in compressed scenarios with exactly two low-momentum same-flavor opposite-charged leptons (electron and muon) in final states and missing transverse momentum $\textbf{p}_{\text{T}}^{\text{miss}}$.
This search uses proton-proton collision data at $\sqrt{s} = 13$~{\TeV} recorded by the ATLAS detector at the Large Hadron Collider (LHC)~\cite{Evans:2008zzb} in 2015 and 2016, corresponding to a total integrated luminosity of 36.1~\ifb.
Figure~\ref{fig:intro_feynman_diagrams} shows the Feynman diagrams representing the electroweakino productions with two leptons final state in association with an initial state radiated jet.
Same-flavor opposite-charged leptons come from the $\widetilde{\chi}^{0}_{2}$ decays in the $\widetilde{\chi}^{0}_{2} \widetilde{\chi}^{\pm}_{1}$ and $\widetilde{\chi}^{0}_{2} \widetilde{\chi}^{0}_{1}$ productions, and from the $\widetilde{\chi}^{\pm}_{1}$ decays in the $\widetilde{\chi}^{\pm}_{1} \widetilde{\chi}^{\mp}_{1}$ production.
The two leptons can be reconstructed in the detector and carry small transverse momentum  \pt.
However, the two LSPs are invisible and back-to-back in the rest frame of their parent electroweakinos.
Because they carry large momentum, the missing transverse energy \met is relatively large.
Similar searches have been performed using $\sqrt{s} = 8$~{\TeV} and $\sqrt{s} = 13$~{\TeV} by the ATLAS~\cite{Aad:2014vma, Aad:2014nua, Aad:2015eda, Aaboud:2016wna} and CMS~\cite{Khachatryan:2014qwa, Khachatryan:2015pot, Sirunyan:2017lae} experiments.
Combining with the results from the LEP experiments, the mass limits for sleptons and charginos are $m(\widetilde{e}_{R}) > 73$~{\GeV}, $m(\widetilde{\mu}_{R}) > 94.6$~{\GeV}, and $m(\widetilde{\chi}^{\pm}_{1}) > 103.5$~{\GeV} or 92.4~{\GeV} depending on the $\Delta m(\widetilde{\chi}^{0}_{1}, \widetilde{\chi}^{\pm}_{1})$.

\begin{figure}[htbp]
    \begin{center}
        \begin{subfigure}[b]{0.32\textwidth}
            \begin{center}
                \includegraphics[scale=1.0]{C1N2-llqqN1N1g-WZ.pdf}
                \caption{The $\widetilde{\chi}^{0}_{2} \widetilde{\chi}^{\pm}_{1}$ production.}
            \end{center}
        \end{subfigure}%
        \begin{subfigure}[b]{0.32\textwidth}
            \begin{center}
                \includegraphics[scale=1.0]{C1C1-llvvN1N1g-WW.pdf}
                \caption{The $\widetilde{\chi}^{\pm}_{1} \widetilde{\chi}^{\mp}_{1}$ production.}
            \end{center}
        \end{subfigure}
        \begin{subfigure}[b]{0.32\textwidth}
            \begin{center}
                \includegraphics[scale=1.0]{N2N1-jllN1N1-Z.pdf}
                \caption{The $\widetilde{\chi}^{0}_{2} \widetilde{\chi}^{0}_{1}$ production.}
            \end{center}
        \end{subfigure}
    \end{center}
    \caption{The Feynman diagrams representing the two leptons final state of (a) $\widetilde{\chi}^{0}_{2} \widetilde{\chi}^{\pm}_{1}$, (b) $\widetilde{\chi}^{\pm}_{1} \widetilde{\chi}^{\mp}_{1}$, (c) $\widetilde{\chi}^{0}_{2} \widetilde{\chi}^{0}_{1}$ productions.}
    \label{fig:intro_feynman_diagrams}
\end{figure}

This dissertation has the following structure.
An introduction is given in Chapter~\ref{chapter:introduction} followed by theoretical foundations in Chapter~\ref{chapter:standard_model} and \ref{chapter:Supersymmetry}.
The experiment facilities are described in Chapter~\ref{chapter:altas_experiment}.
The data and Monte Carlo samples used are detailed in Chapter~\ref{chapter:data}.
Chapter~\ref{chapter:event_reconstruction_and_selection} presents the event reconstruction and the signal region selection.
The background estimation and the systematic uncertainties are discussed in Chapter~\ref{chapter:bkg_estimation}.
The results and interpretation are reported in Chapter~\ref{chapter:results}.
Finally, the conclusions are summarized in Chapter~\ref{chapter:conclusion}.



\chapter{The Standard Model}
\label{chapter:standard_model}
\graphicspath{{figures/standard_model/}}
This chapter outlines the theoretical and mathematical concepts of the high energy particle physics.
The Standard Model of particle physics (SM)~\cite{BF02726525,Glashow,PhysRevLett.19.1264,Herrero:1998eq,CBO9780511791406} is developed since the early 1970s and it has successfully explained almost all experimental results.
The SM is a well-tested and the most successful physics theory to describe the nature of the elementary particles and their interactions.
An overview of the SM is given in Section~\ref{sec:sm}.
After that, an emphasis on the electroweak interactions of the SM is discussed in Section~\ref{sec:ewk}.

%%%
%%%
%%%

\section{The Standard Model of Particle Physics}
\label{sec:sm}
The Standard Model of particle physics is known as the most accurate theory for describing the elementary particles and the interactions between them.
By combining the quantum mechanics and special relativity, the SM is a relativistic \textit{quantum field theory} (QFT) based on a $SU(3)_{C} \otimes SU(2)_{L} \otimes U(1)_{Y}$ symmetry gauge group, where $C$ denotes color, $L$ represents left chirality, and $Y$ stands for weak hypercharge, respectively.
The $SU(3)_{C}$ group is the basis for \textit{Quantum Chromodynamics} (QCD) which describs the strong interaction and the $SU(2)_{L} \otimes U(1)_{Y}$ group is the fundation of the electroweak interaction which unifies the electromatnetic and weak interactions.
Therefore, the SM Lagrangian is invariant under the local gauge transformation.
According to \textit{Noeher's Theorem}~\cite{00411457108231446}, the invariance of an action of a physical system undergoes a symmetry transformation corresponding to a conservation law and vice versa. 
The gauge invariance of the SM Lagrangian corresponds to the conserved quantum numbers, or the charges, of each interaction.
The conserved charges are the three color charge (red, blue, green) for the strong interaction, the third component of the weak isospin $I_{3}^{W}$ for the weak interaction, and the electric charge $Q$ for the electromagnetic interaction.

%%%
%%%
%%%

\subsection{Particle Content}
\label{subsec:sm_particle_content}
According to the SM, all matter around us is made of elementary particles called \textit{quarks} and \textit{leptons}.
The quarks and leptons are called fermions which have half integral spin $s=\frac{1}{2}$, hence the fermions follow the Pauli exclusion principle which says no two fermions have the same quantum state at the same time.
Each fermion has an anti-fermion with the equal mass but carries opposite electric charge, weak isospin and color charge.
There are six quarks and six leptons, they are group into three paris, or "\textit{generations}", ordered by their mass.
The lightest and most stable particles constitute the first generation and they are constituents of ordinary matter.
The heavier and less stable particles form the second and third generations and the heavier particles quickly decay to the next most stable particles.
The three generations of quarks are up ($u$) and down ($d$), charm ($c$) and strange ($s$), and top ($t$) and bottom ($b$) quarks.
The up-type quarks ($u, c, t$) carry $+\frac{2}{3}|e|$ charge and with isospin $+\frac{1}{2}$ while the down-type quarks ($d, s, b$) carry $-\frac{1}{3}|e|$ charge with isospin $-\frac{1}{2}$.
The quarks carry an additional color charge of either red, green, or blue, and hence they only interact via the strong force.
Because the strong force holds quarks together, only non-integer charges of the quark combinations are experimentally allowed.
The quark combinations are called \textit{hadrons} which can be categorised into \textit{mesons} and \textit{baryons}.
The meson is composed by a quark and anti-quark pair ($q\bar{q}$) whereas the baryon is made up by three quarks ($qqq$ or $\bar{q}\bar{q}\bar{q}$).
Only colorless bound states of hadrons are allowed so the quark and anti-quark pair in a meson should contain color and anti-color and the three quarks in a baryon must carry different colors.
The leptons are colorless and are therefore participating in the weak and electromagnetic force only. 
They do not participate in the strong interaction.
The electron-type leptons ($e, \mu, \tau$) carry an elementary charge $|e|$ and their corresponding neutrinos ($\nu_{e}, \nu_{\mu}, \nu_{\tau}$) are neutral.
The neutrinos have very little mass and interact via weak force only.
A summarized table of the properties of quarks and leptons is given in Table~\ref{tab:sm_fermions}.

\begin{table}[htp]
%\begin{center}
\resizebox{\textwidth}{!}{% <------ Don't forget this %
\begin{tabular}{cccccccc}
\hline
\hline
Generation & Fermion & & particle & electric charge $Q$ & weak isospin $I_{3}$ & color charge $C$ & mass [{\GeV}]\\
\hline
\multirow{4}{*}{I} & \multirow{2}{*}{Quark}  & $u$       & up quark          & $+\frac{2}{3}|e|$ & $+\frac{1}{2}$ & r,g,b & 0.0023\\
                   &                         & $d$       & down quark        & $-\frac{1}{3}|e|$ & $-\frac{1}{2}$ & r,g,b & 0.0048\\
                   & \multirow{2}{*}{Lepton} & $e$       & electron          & $-1|e|$           & $-\frac{1}{2}$ & -     & 0.00051\\
                   &                         & $\nu_{e}$ & electron neutrino & 0                 & $+\frac{1}{2}$ & -     & $< 2 \times 10^{-9}$\\
\hline
\multirow{4}{*}{II} & \multirow{2}{*}{Quark}  & $c$         & charm quark   & $+\frac{2}{3}|e|$ & $+\frac{1}{2}$ & r,g,b & 1.275\\
                    &                         & $s$         & strange quark & $-\frac{1}{3}|e|$ & $-\frac{1}{2}$ & r,g,b & 0.095\\
                    & \multirow{2}{*}{Lepton} & $\mu$       & muon          & $-1|e|$           & $-\frac{1}{2}$ & -     & 0.106\\
                    &                         & $\nu_{\mu}$ & muon neutrino & 0                 & $+\frac{1}{2}$ & -     & $< 1.9 \times 10^{-7}$\\
\hline
\multirow{4}{*}{III} & \multirow{2}{*}{Quark}  & $t$          & top quark    & $+\frac{2}{3}|e|$ & $+\frac{1}{2}$ & r,g,b & 173.2\\
                     &                         & $b$          & bottom quark & $-\frac{1}{3}|e|$ & $-\frac{1}{2}$ & r,g,b & 4.18\\
                     & \multirow{2}{*}{Lepton} & $\tau$       & tau          & $-1|e|$           & $-\frac{1}{2}$ & -     & 1.777\\
                     &                         & $\nu_{\tau}$ & tau neutrino & 0                 & $+\frac{1}{2}$ & -     & $< 1.82 \times 10^{-5}$\\

\hline
\hline
\end{tabular}
}
%\end{center}
\caption{The SM fermions with charges and masses~\cite{PDG}.}
\label{tab:sm_fermions}
\end{table}%

%\begin{figure}[htbp]
%\begin{center}
%\includegraphics[scale=0.3]{800px-Standard_Model_of_Elementary_Particles.png}
%\caption{default.
%https://en.wikipedia.org/wiki/Standard_Model
%}
%\label{fig:sm_particles}
%\end{center}
%\end{figure}

%%%
%%%
%%%

Three of the four fundamental forces, strong, weak, electromatnetic forces are parts of the SM now and the gravitational fources could not yet be included in the SM.
Table~\ref{tab:fundamental_forces} shows the four fundamental forces, the relative strength and range together with the theories and the mediators.

\begin{table}[htp]
\begin{center}
\begin{tabular}{ccccc}
\hline
\hline
Force & Rel. Strength & Range (m)& Theory & Mediator\\
\hline
Strong & $10$ & $10^{-15}$ & Chromodynamics & Gluon\\
Weak & $10^{-13}$ & $10^{-18}$ & Flavourdynamics & $W^{\pm}$ and $Z^{0}$ bosons\\
Electromagnetic & $10^{-2}$ & $\infty$ & Electrodynamics & Photon\\
\hline
Gravitational & $10^{-42}$ & $\infty$ & General relativity & Graviton\\
\hline
\hline
\end{tabular}
\end{center}
\caption{The four fundamental forces with the relative strength, interaction range, describing theory, and the mediator.}
\label{tab:fundamental_forces}
\end{table}%

%%%
%%%
%%%

























% Forces and carrier particles

% There are four fundamental forces at work in the universe: the strong force, the weak force, the electromagnetic force, and the gravitational force. They work over different ranges and have different strengths. Gravity is the weakest but it has an infinite range. The electromagnetic force also has infinite range but it is many times stronger than gravity. The weak and strong forces are effective only over a very short range and dominate only at the level of subatomic particles. Despite its name, the weak force is much stronger than gravity but it is indeed the weakest of the other three. The strong force, as the name suggests, is the strongest of all four fundamental interactions.
% Three of the fundamental forces result from the exchange of force-carrier particles, which belong to a broader group called “bosons”. Particles of matter transfer discrete amounts of energy by exchanging bosons with each other. Each fundamental force has its own corresponding boson – the strong force is carried by the “gluon”, the electromagnetic force is carried by the “photon”, and the “W and Z bosons” are responsible for the weak force. Although not yet found, the “graviton” should be the corresponding force-carrying particle of gravity. The Standard Model includes the electromagnetic, strong and weak forces and all their carrier particles, and explains well how these forces act on all of the matter particles. However, the most familiar force in our everyday lives, gravity, is not part of the Standard Model, as fitting gravity comfortably into this framework has proved to be a difficult challenge. The quantum theory used to describe the micro world, and the general theory of relativity used to describe the macro world, are difficult to fit into a single framework. No one has managed to make the two mathematically compatible in the context of the Standard Model. But luckily for particle physics, when it comes to the minuscule scale of particles, the effect of gravity is so weak as to be negligible. Only when matter is in bulk, at the scale of the human body or of the planets for example, does the effect of gravity dominate. So the Standard Model still works well despite its reluctant exclusion of one of the fundamental forces.
% So far so good, but...

% ...it is not time for physicists to call it a day just yet. Even though the Standard Model is currently the best description there is of the subatomic world, it does not explain the complete picture. The theory incorporates only three out of the four fundamental forces, omitting gravity. There are also important questions that it does not answer, such as “What is dark matter?”, or “What happened to the antimatter after the big bang?”, “Why are there three generations of quarks and leptons with such a different mass scale?” and more. Last but not least is a particle called the Higgs boson, an essential component of the Standard Model.
% On 4 July 2012, the ATLAS and CMS experiments at CERN's Large Hadron Collider (LHC) announced they had each observed a new particle in the mass region around 126 GeV. This particle is consistent with the Higgs boson but it will take further work to determine whether or not it is the Higgs boson predicted by the Standard Model. The Higgs boson, as proposed within the Standard Model, is the simplest manifestation of the Brout-Englert-Higgs mechanism. Other types of Higgs bosons are predicted by other theories that go beyond the Standard Model.
% On 8 October 2013 the Nobel prize in physics was awarded jointly to François Englert and Peter Higgs "for the theoretical discovery of a mechanism that contributes to our understanding of the origin of mass of subatomic particles, and which recently was confirmed through the discovery of the predicted fundamental particle, by the ATLAS and CMS experiments at CERN's Large Hadron Collider."
% So although the Standard Model accurately describes the phenomena within its domain, it is still incomplete. Perhaps it is only a part of a bigger picture that includes new physics hidden deep in the subatomic world or in the dark recesses of the universe. New information from experiments at the LHC will help us to find more of these missing pieces.



\chapter{Suppersymmetry}
\label{chapter:Suppersymmetry}
\graphicspath{{figures/Suppersymmetry/}}
\input{chapter_Suppersymmetry}


\chapter{The ATLAS Experiment at LHC}
\label{chapter:altas_experiment}
\graphicspath{{figures/atlas_experiment/}}
The European Organization for Nuclear Research (CERN\footnote{The name CERN is derived from the acronym for the French Conseil Europ\'{e}en pour la Recherch Nucl\'{e}aire}) was founded in 1954 and is based in the suburb of Geneva on the Franco\textendash Swiss border.
The main function of CERN is to provide particle accelerators and detectors for high-energy physics research.
The physicists and engineers at CERN are probing the fundamental structure of the universe using the world's largest and most complex scientific facility \textemdash \ the Large Hadron Collider (LHC)~\cite{1748-0221-3-08-S08001}.
In the LHC, the particles are boosted to high energies and collide at close to the speed of light.
The results of the collisions are recorded by the various detectors.
There are seven experiments at the LHC.
The biggest of these experiments are ATLAS (A Toroidal LHC ApparatuS)~\cite{1748-0221-3-08-S08003} and CMS (Compact Muon Solenoid)~\cite{1748-0221-3-08-S08004} which use general-purpose detectors to investigate a broad physics programme ranging from the search for the Higgs boson to extra dimensions and particles that could make up dark matter.
The ALICE (A Large Ion Collider Experiment)~\cite{1748-0221-3-08-S08002} experiment is designed to study the physics of quark-gluon plasma form and the LHCb (Large Hadron Collider beauty)~\cite{1748-0221-3-08-S08005} experiment specializes in investigating of CP violation by studying the $b$-quark.
These four detectors sit underground in huge caverns of the LHC ring.
The rest three experiments, TOTEM~\cite{1748-0221-3-08-S08007}, LHCf~\cite{1748-0221-3-08-S08006}, and MoEDAL~\cite{Pinfold:1181486}, are smaller.
The TOTEM (TOTal Elastic and diffractive cross section Measurement)~\cite{1748-0221-3-08-S08007} experiment aims at the measurement of total cross section, elastic scattering, and diffractive dissociation.
The LHCf (Large Hadron Collider forward)~\cite{1748-0221-3-08-S08006} experiment is intended to measure the neutral particle produced by the collider using the forward particles.
The prime motivation of the MoEDAL (Monopole and Exotics Detector at the LHC)~\cite{Pinfold:1181486} experiment is to search directly for the magnetic monopole.

\section{The Large Hadron Collide}
The LHC~\cite{1748-0221-3-08-S08001} is the world's largest and most powerful accelerator which accelerates and collides protons in a 26.7 km circumference crossing the Franco\textendash Swiss border 100 m underground.
The LHC is designed for collisions at a centre-of-mass energy $\sqrt{s}=14$ TeV and an instantaneous luminosity of $\mathcal{L} =10^{34} \ \textrm{cm}^{-2}\textrm{s}^{-1}$.
The first beam was circulated through the collider on the morning of 10 September 2008~\cite{CERN-COURIER-Sep192008}.
However, a magnet quench incident occurred on 19 September 2008 and caused extensive damage to over 50 superconducting magnets, their mountings, and the vacuum pipe.
Most of 2009 was spent on repairs the damage caused by the magnet quench incident and the operations resumed on 20 November of that year.
The first phase of data-taking (Run 1) started at the end of 2009 and the beam energy was increased to a centre-of-mass $\sqrt{s}=7$ TeV in 2011 and $\sqrt{s} = 8$ TeV in 2012.
The total integrated luminosity of 5.46 fb$^{-1}$ was collected in 2011 and of 22.8 fb$^{-1}$ was collected in 2012.
Since 13 February 2013 the LHC was in the Long Shutdown 1 (LS1) phase for maintenance and upgrades.
On 5 April 2015, the LHC restarted and was operating at a centre-of-mass energy $\sqrt{s}=13$ TeV throughout the Run 2 phase (2015 to 2017).



\section{The ATLAS experiment}

\subsection{The Inner Detector}

\subsection{The Calorimeter}

\subsection{The Muon Spectrometer}

\subsection{The Trigger system}

\subsection{}


%\chapter{The Electron Isolation Efficiency and Scale Factors}
%\label{chapter:electron_isolation}
%\graphicspath{{figures/electron_isolation}}
%\input{chapter_electron_isolation}


%\chapter{The Real Lepton Efficiencies}
%\label{chapter:real_lepton_efficiencies}
%\graphicspath{{figures/real_lepton_efficiencies}}
%\input{chapter_real_lepton_efficiencies}

%\chapter{Results}
%\label{chapter:results}
%\graphicspath{{figures/results/}}
%%\chapter{Conclusion}
\label{conc}
\begin{quote}
\raggedright{\emph{``We must not forget that when radium was discovered no one knew that it would prove useful in hospitals. The work was one of pure science. And this is a proof that scientific work must not be considered from the point of view of the direct usefulness of it. It must be done for itself, for the beauty of science, and then there is always the chance that a scientific discovery may become like the radium a benefit for humanity.''}} \\
%\raggedright{- Marie Curie (1867 - 1934), Lecture at Vassar College, May 14, 1921}
\raggedleft{- Marie Curie (1867 - 1934)}
\end{quote}
First, a search for a high-mass Higgs boson in the \hwwlnqq\ decay channel was performed using 20.3 fb$^{-1}$ of LHC $pp$ collision data recorded by the ATLAS detector at a center-of-mass energy of \ensuremath{\sqrt{s} = 8}\TeV. No significant deviation from the SM background-only prediction is observed. Thus, for both ggF and VBF production modes, upper limits on $\sigma_H \times \text{BR}$(\hww) are set, as a function the Higgs mass \mH, in three different signal width scenarios of a high-mass Higgs boson with a narrow width, an intermediate width, and a SM width. The mass range of the derived limits is $300\GeV \leq \mH \leq 1000\GeV$, with an extension up to 1500\GeV\ for the narrow-width scenario. 

A second, more model-independent search was performed in the same decay channel using 3.2 fb$^{-1}$ of ATLAS recorded data from the upgraded LHC with $pp$ collisions at a center-of-mass energy of \ensuremath{\sqrt{s} = 13}\TeV. The signal widths tested in this search include the previous narrow-width as well as three new intermediate widths at 5, 10, and 15\% of \mH. Again, no significant deviations from the background-only hypothesis are observed, leading to upper limits on the $\sigma_H \times \text{BR}$(\hww) for the different signal width scenarios. The mass range of the limits is substantially improved, in regards to the previous search, and extends up to 3000\GeV. 

The results from both searches are substantial improvements over the previous results from the ATLAS experiment in terms of both the cross-section times branching ratio values excluded and the mass range explored.

Searches in this decay channel, \wwlnqq, are still alive and active! The scalar results presented in Chapter~\ref{chap:13tev} are included in the recently submitted paper~\cite{13TeV_combo_paper}, which combines searches for heavy narrow-width resonances decaying to $WW$, $WZ$, and $ZZ$ with final states $\nu\nu qq$, $\ell\nu qq$, $\ell\ell qq$, and $qqqq$. Also, in the course of writing this dissertation, ATLAS has already recorded another 15$\text{ fb}^{-1}$ of data at $\sqrt{s} = 13\TeV$! Analysis of the new data is already underway in this channel, adding more data to the previous results and looking into VBF production and the resolved regime.

%\FloatBarrier

%\chapter{Conclusion}
%\label{chapter:conclusion}
%\graphicspath{{figures/conclusion/}}
%%\chapter{Conclusion}
\label{conc}
\begin{quote}
\raggedright{\emph{``We must not forget that when radium was discovered no one knew that it would prove useful in hospitals. The work was one of pure science. And this is a proof that scientific work must not be considered from the point of view of the direct usefulness of it. It must be done for itself, for the beauty of science, and then there is always the chance that a scientific discovery may become like the radium a benefit for humanity.''}} \\
%\raggedright{- Marie Curie (1867 - 1934), Lecture at Vassar College, May 14, 1921}
\raggedleft{- Marie Curie (1867 - 1934)}
\end{quote}
First, a search for a high-mass Higgs boson in the \hwwlnqq\ decay channel was performed using 20.3 fb$^{-1}$ of LHC $pp$ collision data recorded by the ATLAS detector at a center-of-mass energy of \ensuremath{\sqrt{s} = 8}\TeV. No significant deviation from the SM background-only prediction is observed. Thus, for both ggF and VBF production modes, upper limits on $\sigma_H \times \text{BR}$(\hww) are set, as a function the Higgs mass \mH, in three different signal width scenarios of a high-mass Higgs boson with a narrow width, an intermediate width, and a SM width. The mass range of the derived limits is $300\GeV \leq \mH \leq 1000\GeV$, with an extension up to 1500\GeV\ for the narrow-width scenario. 

A second, more model-independent search was performed in the same decay channel using 3.2 fb$^{-1}$ of ATLAS recorded data from the upgraded LHC with $pp$ collisions at a center-of-mass energy of \ensuremath{\sqrt{s} = 13}\TeV. The signal widths tested in this search include the previous narrow-width as well as three new intermediate widths at 5, 10, and 15\% of \mH. Again, no significant deviations from the background-only hypothesis are observed, leading to upper limits on the $\sigma_H \times \text{BR}$(\hww) for the different signal width scenarios. The mass range of the limits is substantially improved, in regards to the previous search, and extends up to 3000\GeV. 

The results from both searches are substantial improvements over the previous results from the ATLAS experiment in terms of both the cross-section times branching ratio values excluded and the mass range explored.

Searches in this decay channel, \wwlnqq, are still alive and active! The scalar results presented in Chapter~\ref{chap:13tev} are included in the recently submitted paper~\cite{13TeV_combo_paper}, which combines searches for heavy narrow-width resonances decaying to $WW$, $WZ$, and $ZZ$ with final states $\nu\nu qq$, $\ell\nu qq$, $\ell\ell qq$, and $qqqq$. Also, in the course of writing this dissertation, ATLAS has already recorded another 15$\text{ fb}^{-1}$ of data at $\sqrt{s} = 13\TeV$! Analysis of the new data is already underway in this channel, adding more data to the previous results and looking into VBF production and the resolved regime.

%------------------------------------------------------------------------------- 
\clearpage

%\appendix
%\part*{Appendix}
%\addcontentsline{toc}{part}{Appendix}

%\section{Simulated samples, cross-sections and equivalent luminosities}
%\label{app:samples}
%The Monte Carlo (MC) simulated samples are used to model the SUSY signals and to estimate the SM background.
The MC samples are processed using a ATLAS detector full simulation or a fast simulation based on {\GEANT}4~\cite{} simulation package.
The full simulation simulates the detailed properties of the ATLAS detector while the fast simulation uses a parameterized calorimeter response and simulates ID and MS based on {\GEANT}4.

%%%
%%%
%%%

\section{Samples used for strong interaction}
\label{sec:app_samples_strong}

Table~\ref{tab:app_sample_strong} shows the event generator, parton shower, cross-section normalization, PDF set, and the set of tunned parameters for modelling for all samples.

\begin{table}[htp]
%\begin{center}
\resizebox{\textwidth}{!}{% <------ Don't forget this %
\begin{tabular}{cccccc}
\hline
\hline
Physics process & Event generator & Parton shower & Cross-section normalization & PDF set & Set of tunned parameters\\
\hline
\hline
$t\bar{t}+X$\\
$t\bar{t}W, t\bar{t}Z/\gamma^{*}$ & MG5\_{\scriptsize A}MC@NLO 2.2.2 & {\PYTHIA} 8.186 & NLO & NNPDF2.3LO & A14\\
$t\bar{t}H$ & MG5\_{\scriptsize A}MC@NLO 2.3.2 & {\PYTHIA} 8.186 & NLO & NNPDF2.3LO & A14\\
4$t$ & MG5\_{\scriptsize A}MC@NLO 2.2.2 & {\PYTHIA} 8.186 & NLO & NNPDF2.3LO & A14\\
Dibosno\\
$ZZ, WZ$ & {\SHERPA} 2.2.1 & {\SHERPA} 2.2.1 & NLO & NNPDF2.3LO & {\SHERPA} default\\
Other (inc. $W^{\pm}W^{\pm}$) & {\SHERPA} 2.1.1 & {\SHERPA} 2.1.1 & NLO & CT10 & {\SHERPA} default\\
Rare\\
$t\bar{t}WW, t\bar{t}WZ$ & MG5\_{\scriptsize A}MC@NLO 2.2.2 & {\PYTHIA} 8.186 & NLO & NNPDF2.3LO & A14\\
$tZ, tWZ, tt\bar{t}$  & MG5\_{\scriptsize A}MC@NLO 2.2.2 & {\PYTHIA} 8.186 & LO & NNPDF2.3LO & A14\\
$WH, ZH$ & MG5\_{\scriptsize A}MC@NLO 2.2.2 & {\PYTHIA} 8.186 & NLO & NNPDF2.3LO & A14\\
Triboson & {\SHERPA} 2.1.1 & {\SHERPA} 2.1.1 & NLO & CT10 & {\SHERPA} default\\
\hline
\hline
\end{tabular}
%\end{center}
}
\caption{}
\label{tab:app_sample_strong}
\end{table}%


%%%
%%%
%%%

\section{Samples used for weak interaction}
\label{sec:app_samples}

%%%
%%%
%%%
%\clearpage
%-------------------------------------------------------------------------------

\clearpage

%\printbibliography
\clearpage










\bibliographystyle{unsrt}
%% You need a file named `outhesis_references.bib' to use BibTex here
%\bibliography{outhesis_references}
\bibliography{bib_standard_model,bib_atlas_experiment}
%\bibliography{}



% \appendix{A}
% \input{appendices/appendixA}
%\begin{appendices}
%\input{chapter_hardware}
%\input{chapter_auxiliaryPlots8TeV}
%\input{chapter_JSSuncSmoothing}
%\end{appendices}


% \backmatter


\end{document}

