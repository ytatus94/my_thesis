The ATLAS and CMS collaborations announced the discovery of the Higgs boson in July 2012, completing the particle content in the Standard Model.
Although the Standard Model gets a great triumph, it is not considered to be the complete theory of particle physics.
While the physicists try to search for new physics beyond the Standard Model (SM), several new theories have been developed.
Among the newly developed theories, Supersymmetry (SUSY) is one of the most favorite ones.
SUSY predicts the existence of supersymmetric partner particles and it is the only possible extension of the space-time symmetry of particle interactions.
There are supersymmetric particles associated with each SM particle where the spin differs by 1/2.
This dissertation focuses on searching for electroweak production of supersymmetric particles with compressed mass spectra in the final states with exactly two low-momentum leptons and missing transverse momentum.
The proton-proton collision data is recorded by the ATLAS detector at the Large Hadron Collider in 2015 to 2016 corresponding to 36.1~{\ifb} of integrated luminosity at $\sqrt{s} = 13$~{\TeV}.
Events with same-flavor and opposite electric charge lepton pairs are selected.
The data are found to be consistent with the Standard Model prediction.
Results are interpreted using non-universal Higgs mass model with two extra parameters (NUHM2) with small mass differences between the masses of produced supersymmetric particles.
Upper limits of the cross-section at 95\% confidence level are set for the NUHM2 model as a function of the universal gaugino mass $m_{1/2}$.
