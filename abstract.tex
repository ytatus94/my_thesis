Th ATLAS and CMS collaborations announced the discovery of the Higgs boson like new particle in July 2012, completing of particle content in the Standard Model.
Although the Standard Model gets a great triumph, it is not considered as the fundamental theory of particle physics.
While the physicists try to search the new physics beyond the Standard Model, several new theories have been developed.
Among the newly developed theories, the Supersymmetry (SUSY) is the most favorited one.
The SUSY predicts the existence of the supersymmetric particles and it is the only possible extensions of the space-time symmetries of the particle interactions.
There is a supersymmetric particles associated with each SM particle where the spin differs by 1/2.
This thesis is focus on searching the supersymmetric particles with compressed mass spectra in the final states with exactly two leptons and missing transverse momentum.
The proton-proton collision data is recorded by the ATLAS detector at the Large Hadron Collider in 2015 to 2016 corresponding to 36.1~{\ifb} of integrated luminosity at $\sqrt{s} = 13$~{\TeV}.

An introduction is given in the Chapter~\ref{chapter:introduction} followed by the theoretical foundations in the Chapter~\ref{chapter:standard_model} and \ref{chapter:Suppersymmetry}. The experiment facilities are described in Chapter~\ref{chapter:altas_experiment}.
