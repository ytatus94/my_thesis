The ATLAS and CMS collaborations announced the discovery of the Higgs boson like new particle in July 2012, completing of particle content in the Standard Model.
Although the Standard Model gets a great triumph, it is not considered as the fundamental theory of particle physics.
While the physicists try to search the new physics beyond the Standard Model, several new theories have been developed.
Among the newly developed theories, the Supersymmetry (SUSY) is the most favorite one.
The SUSY predicts the existence of the supersymmetric particles and it is the only possible extensions of the space-time symmetries of the particle interactions.
There are supersymmetric particles associated with each SM particle where the spin differs by 1/2.
This dissertation focus on searching the electroweak production of supersymmetric particles with compressed mass spectra in the final states with exactly two low-momentum leptons and missing transverse momentum.
The proton-proton collision data is recorded by the ATLAS detector at the Large Hadron Collider in 2015 to 2016 corresponding to 36.1~{\ifb} of integrated luminosity at $\sqrt{s} = 13$~{\TeV}.
Events with same-flavor and opposite electric charge lepton pairs are selected.
The data are found to be consistent with the Standard Model prediction.
Results are interpreted using NUHM2 model with small mass difference between the masses produced supersymmetric particles and the lightest neutralino.
Upper limits of the cross-section at 95\% confidence level are set for the NUHM2 as a function of the universal gaugino mass $m_{1/2}$.
