%\chapter{Conclusion}
\label{conc}
\begin{quote}
\raggedright{\emph{``We must not forget that when radium was discovered no one knew that it would prove useful in hospitals. The work was one of pure science. And this is a proof that scientific work must not be considered from the point of view of the direct usefulness of it. It must be done for itself, for the beauty of science, and then there is always the chance that a scientific discovery may become like the radium a benefit for humanity.''}} \\
%\raggedright{- Marie Curie (1867 - 1934), Lecture at Vassar College, May 14, 1921}
\raggedleft{- Marie Curie (1867 - 1934)}
\end{quote}
First, a search for a high-mass Higgs boson in the \hwwlnqq\ decay channel was performed using 20.3 fb$^{-1}$ of LHC $pp$ collision data recorded by the ATLAS detector at a center-of-mass energy of \ensuremath{\sqrt{s} = 8}\TeV. No significant deviation from the SM background-only prediction is observed. Thus, for both ggF and VBF production modes, upper limits on $\sigma_H \times \text{BR}$(\hww) are set, as a function the Higgs mass \mH, in three different signal width scenarios of a high-mass Higgs boson with a narrow width, an intermediate width, and a SM width. The mass range of the derived limits is $300\GeV \leq \mH \leq 1000\GeV$, with an extension up to 1500\GeV\ for the narrow-width scenario. 

A second, more model-independent search was performed in the same decay channel using 3.2 fb$^{-1}$ of ATLAS recorded data from the upgraded LHC with $pp$ collisions at a center-of-mass energy of \ensuremath{\sqrt{s} = 13}\TeV. The signal widths tested in this search include the previous narrow-width as well as three new intermediate widths at 5, 10, and 15\% of \mH. Again, no significant deviations from the background-only hypothesis are observed, leading to upper limits on the $\sigma_H \times \text{BR}$(\hww) for the different signal width scenarios. The mass range of the limits is substantially improved, in regards to the previous search, and extends up to 3000\GeV. 

The results from both searches are substantial improvements over the previous results from the ATLAS experiment in terms of both the cross-section times branching ratio values excluded and the mass range explored.

Searches in this decay channel, \wwlnqq, are still alive and active! The scalar results presented in Chapter~\ref{chap:13tev} are included in the recently submitted paper~\cite{13TeV_combo_paper}, which combines searches for heavy narrow-width resonances decaying to $WW$, $WZ$, and $ZZ$ with final states $\nu\nu qq$, $\ell\nu qq$, $\ell\ell qq$, and $qqqq$. Also, in the course of writing this dissertation, ATLAS has already recorded another 15$\text{ fb}^{-1}$ of data at $\sqrt{s} = 13\TeV$! Analysis of the new data is already underway in this channel, adding more data to the previous results and looking into VBF production and the resolved regime.