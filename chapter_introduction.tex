\begin{quote}
\raggedright{\emph{``There is nothing like looking, if you want to find something. You certainly usually find something, if you look, but it is not always quite the something you were after.''}} \\
\raggedleft{- J.R.R. Tolkien, The Hobbit}
\end{quote}
It has been a driving question throughout the history of science: What are the fundamental constituents of matter and how do they interact with one another? Particle physics addresses this question directly. The contributions of many physicists over the decades, has culminated in an impressively descriptive and predictive theory of the fundamental constituents of matter and their interactions, referred to as the Standard Model. 

The recent discovery of the Higgs boson, as predicted by the Standard Model, is a pinnacle moment in the validation of the particle physics theory. However, there are still many compelling reasons to search for physics beyond the Standard Model, particularly for additional higher mass scalar (Higgs-like) bosons. 

This dissertation outlines these motivations in Chapter~\ref{chap_theory}, after a description of the Standard Model. The ATLAS detector and Large Hadron Collider, used to search for the Higgs bosons, are described in Chapter~\ref{chap_detector}, preceding the description of two searches at different center-of-mass collision energies $\sqrt{s} = 8\TeV$ and 13\TeV\ in Chapters~\ref{chap_analysis8TeV} and~\ref{chap:13tev}, respectively. Both are searches for a high-mass Higgs boson in the decay channel $H~{\rightarrow}~WW~{\rightarrow}~\ell\nu{qq'}$ ($qq'$ referred to hereafter as simply $qq$). Finally, Chapter~\ref{conc} discusses the conclusions from both analyses, which offer substantial improvements over the results of previous searches. 