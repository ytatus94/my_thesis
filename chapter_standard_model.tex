This chapter outlines the theoretical and mathematical concepts of the high energy particle physics.
The Standard Model of particle physics (SM)~\cite{BF02726525,Glashow,PhysRevLett.19.1264,Herrero:1998eq,CBO9780511791406} is developed since the early 1970s and it has successfully explained almost all experimental results.
The SM is a well-tested and the most successful physics theory to describe the nature of the elementary particles and their interactions.
An overview of the SM is given in Section~\ref{sec:sm}.
After that, some of the open questions are mentioned in Section~\ref{sec:sm_bsm}.

%%%
%%%
%%%

\section{The Standard Model of Particle Physics}
\label{sec:sm}
The Standard Model of particle physics is known as the most accurate theory for describing the elementary particles and the interactions between them.
By combining the quantum mechanics and special relativity, the SM is a relativistic \textit{Quantum Field Theory} (QFT) based on a $SU(3)_{C} \otimes SU(2)_{L} \otimes U(1)_{Y}$ symmetry gauge group, where $C$ denotes color, $L$ represents left chirality, and $Y$ stands for weak hypercharge, respectively.
The $SU(3)_{C}$ group is the basis for \textit{Quantum Chromodynamics} (QCD) which describs the strong interaction and the $SU(2)_{L} \otimes U(1)_{Y}$ group is the fundation of the electroweak interaction which unifies the electromatnetic and weak interactions.
Therefore, the SM Lagrangian is invariant under the local gauge transformation.
According to \textit{Noeher's Theorem}~\cite{00411457108231446}, the invariance of an action of a physical system undergoes a symmetry transformation corresponding to a conservation law and vice versa. 
The gauge invariance of the SM Lagrangian corresponds to the conserved quantum numbers, or the charges, of each interaction.
The conserved charges are the three color charge (red, blue, green) for the strong interaction, the third component of the weak isospin $I_{3}$ for the weak interaction, and the electric charge $Q$ for the electromagnetic interaction.

%%%
%%%
%%%

\subsection{Particle Content}
\label{subsec:sm_particle_content}
According to the SM, all matter around us is made of elementary particles called \textit{quarks} and \textit{leptons}.
The quarks and leptons are called fermions which have half integral spin $s=\frac{1}{2}$, hence the fermions follow the Pauli exclusion principle which says no two fermions have the same quantum state at the same time.
Each fermion has an anti-fermion with the equal mass but carries opposite electric charge, weak isospin and color charge.
There are six quarks and six leptons, they are group into three paris, or "\textit{generations}", ordered by their mass.
The lightest and most stable particles constitute the first generation and they are constituents of ordinary matter.
The heavier and less stable particles form the second and third generations and the heavier particles quickly decay to the next most stable particles.
The three generations of quarks are up ($u$) and down ($d$), charm ($c$) and strange ($s$), and top ($t$) and bottom ($b$) quarks.
The up-type quarks ($u, c, t$) carry $+\frac{2}{3}|e|$ charge and with isospin $+\frac{1}{2}$ while the down-type quarks ($d, s, b$) carry $-\frac{1}{3}|e|$ charge with isospin $-\frac{1}{2}$.
The quarks carry an additional color charge of either red, green, or blue, and hence they only interact via the strong force.
Because the strong force holds quarks together, only non-integer charges of the quark combinations are experimentally allowed.
The quark combinations are called \textit{hadrons} which can be categorised into \textit{mesons} and \textit{baryons}.
The meson is composed by a quark and anti-quark pair ($q\bar{q}$) whereas the baryon is made up by three quarks ($qqq$ or $\bar{q}\bar{q}\bar{q}$).
Only colorless bound states of hadrons are allowed so the quark and anti-quark pair in a meson should contain color and anti-color and the three quarks in a baryon must carry different colors.
The leptons are colorless and are therefore participating in the weak and electromagnetic force only. 
They do not participate in the strong interaction.
The electron-type leptons ($e, \mu, \tau$) carry an elementary charge $|e|$ and their corresponding neutrinos ($\nu_{e}, \nu_{\mu}, \nu_{\tau}$) are neutral.
The neutrinos have very little mass and interact via weak force only.
A summarized table of the properties of quarks and leptons is given in Table~\ref{tab:sm_fermions}.

\begin{table}[htp]
%\begin{center}
\resizebox{\textwidth}{!}{% <------ Don't forget this %
\begin{tabular}{cccccccc}
\hline
\hline
Generation & Fermion & & particle & electric charge $Q$ & weak isospin $I_{3}$ & color charge $C$ & mass [{\GeV}]\\
\hline
\multirow{4}{*}{I} & \multirow{2}{*}{Quark}  & $u$       & up quark          & $+\frac{2}{3}|e|$ & $+\frac{1}{2}$ & r,g,b & 0.0023\\
                   &                         & $d$       & down quark        & $-\frac{1}{3}|e|$ & $-\frac{1}{2}$ & r,g,b & 0.0048\\
                   & \multirow{2}{*}{Lepton} & $e$       & electron          & $-1|e|$           & $-\frac{1}{2}$ & -     & 0.00051\\
                   &                         & $\nu_{e}$ & electron neutrino & 0                 & $+\frac{1}{2}$ & -     & $< 2 \times 10^{-9}$\\
\hline
\multirow{4}{*}{II} & \multirow{2}{*}{Quark}  & $c$         & charm quark   & $+\frac{2}{3}|e|$ & $+\frac{1}{2}$ & r,g,b & 1.275\\
                    &                         & $s$         & strange quark & $-\frac{1}{3}|e|$ & $-\frac{1}{2}$ & r,g,b & 0.095\\
                    & \multirow{2}{*}{Lepton} & $\mu$       & muon          & $-1|e|$           & $-\frac{1}{2}$ & -     & 0.106\\
                    &                         & $\nu_{\mu}$ & muon neutrino & 0                 & $+\frac{1}{2}$ & -     & $< 1.9 \times 10^{-7}$\\
\hline
\multirow{4}{*}{III} & \multirow{2}{*}{Quark}  & $t$          & top quark    & $+\frac{2}{3}|e|$ & $+\frac{1}{2}$ & r,g,b & 173.2\\
                     &                         & $b$          & bottom quark & $-\frac{1}{3}|e|$ & $-\frac{1}{2}$ & r,g,b & 4.18\\
                     & \multirow{2}{*}{Lepton} & $\tau$       & tau          & $-1|e|$           & $-\frac{1}{2}$ & -     & 1.777\\
                     &                         & $\nu_{\tau}$ & tau neutrino & 0                 & $+\frac{1}{2}$ & -     & $< 1.82 \times 10^{-5}$\\

\hline
\hline
\end{tabular}
}
%\end{center}
\caption{The SM fermions with charges and masses~\cite{PDG}.}
\label{tab:sm_fermions}
\end{table}%

%\begin{figure}[htbp]
%\begin{center}
%\includegraphics[scale=0.3]{800px-Standard_Model_of_Elementary_Particles.png}
%\caption{default.
%https://en.wikipedia.org/wiki/Standard_Model
%}
%\label{fig:sm_particles}
%\end{center}
%\end{figure}

There are four fundamental forces in the universe: the strong force, the weak force, the electromagnetic force, and the gravitational force.
The first three forces are described in the SM, however, the gravitational force could not yet be included in the SM.
Because the effect of the gravitational force is very weak and can be negligible, the SM works well without considering the gravitational force.
Each force has a force-carrier particle called gauge boson and there is a quantum number associate to it.
The gauge bosons of the strong force are eight massless \textit{gluons}, $g$, which associate to color charge $C$.
The gauge bosons of the weak force are $W^{\pm}$ and $Z^{0}$ bosons which associate to weak isospin $I_{3}$.
The gauge boson of the electromatnetic force is massless \textit{photon}, $\gamma$, which associates to electric charge $Q$.
Although the gluon and photon are massless particles, the $W^{\pm}$ and $Z^{0}$ bosons are massive.
The mass of the $W^{\pm}$ and $Z^{0}$ bosons are $m_{W}=80.385 \pm 0.015$~{\GeV} and $m_{Z}=91.1876 \pm 0.0021$~{\GeV}~\cite{PDG}, respectively.
Table~\ref{tab:sm_fundamental_forces} shows the four fundamental forces, the relative strength and range together with the theories and the mediators.

\begin{table}[htp]
%\begin{center}
\resizebox{\textwidth}{!}{% <------ Don't forget this %
\begin{tabular}{cccccc}
\hline
\hline
Force & Rel. Strength & Range [m]& Theory & Mediator & Mass [{\GeV}]\\
\hline
Strong & $10$ & $10^{-15}$ & Chromodynamics & Gluon & 0\\
Weak & $10^{-13}$ & $10^{-18}$ & Flavourdynamics & $W^{\pm}$ and $Z^{0}$ bosons & 80.4/91.2\\
Electromagnetic & $10^{-2}$ & $\infty$ & Electrodynamics & Photon & 0\\
\hline
Gravitational & $10^{-42}$ & $\infty$ & General relativity & Graviton & -\\
\hline
\hline
\end{tabular}
}
%\end{center}
\caption{The four fundamental forces with the relative strength, interaction range, describing theory, and the mediator with its mass.
The gravitational force is not a part of the SM and the graviton is a theoretical particle.}
\label{tab:sm_fundamental_forces}
\end{table}%

%%%
%%%
%%%

\subsection{Local Gauge Theory}
\label{subsec:sm_gauge_theory}
The Lagrangian density of the SM for the free fields listed in the Equation~\ref{eq:sm_lagrangian} is invariant under local gauge transformation.
\begin{equation}
\mathcal{L} = \bar{\psi}(i\gamma^{\mu}\partial_{\mu} - m)\psi + e\bar{\psi}\gamma^{\mu}\psi\bm{A}_{\mu} - \frac{1}{4}\bm{F}_{\mu\nu}\bm{F}^{\mu\nu}
\label{eq:sm_lagrangian}
\end{equation}
where $\bm{F}_{\mu\nu} = \partial_{\mu}\bm{A}_{\nu} - \partial_{\nu}\bm{A}_{\mu}$.
The local gauge transformation means the scalar field $\psi$ and the vector field $\bm{A}_{\mu}$ transform as
\begin{align}
\psi(x) & \rightarrow \psi'(x) = e^{i\theta(x)}\psi(x)\\
\bm{A}_{\mu}(x) & \rightarrow \bm{A}_{\mu}'(x) = \bm{A}_{\mu}(x) + \frac{1}{e}\partial_{\mu}\theta(x)
\label{eq:sm_gauge_transformation}
\end{align}
By intruducing the gauge term, i.e. the vector field, the interacting force can be obtained by calculating the derivatives of the \textit{Euler-Lagrange equations}.
The gauge field can be associated to particular spin one gauge bosons which mediate the force.
The number of the mediating gauge bosons is equal to the dimension of the symmetry group.
From the group theory, the dimension of an unitary group $U(n)$ is $n^{2}$ and the dimension of a special unitary group $SU(n)$ is $n^{2} - 1$.
Because the SM is based on a $SU(3)_{C} \otimes SU(2)_{L} \otimes U(1)_{Y}$ symmetry gauge group, the number of mediators are 8 for $SU(3)$, 3 for $SU(2)$, and 1 for $U(1)$ corresponding to 8 gluons for the strong interaction, 3 gauge bosons ($W^{\pm}$ and $Z^{0}$) for weak interaction, and 1 photon for the electromagnetic interaction.


%%%
%%%
%%%

\subsection{Strong interaction}
\label{subsec:sm_strong_interaction}

%%%
%%%
%%%

\subsection{Electroweak interaction}
\label{subsec:sm_ewk_interaction}

%%%
%%%
%%%

\section{Beyond the Standard Model}
\label{sec:sm_bsm}
Although the SM is an incredible successful theory for explaining the phenomenon in the particle physics, it leaves some questions which can no be answered.
Some of the unanswered questions are introduced in the rest part of this section.

%%%
%%%
%%%

\subsection{Hierarchy problem}
\label{subsec:sm_hierarchy_problem}
The weakest force in the SM is the weak force but the strength of the weak force is $10^{24}$ times as strong as gravitational force which doesn't incorporate into the SM.
The large discrepancy between the weak force and the gravitational force is called the hierarchy problem~\cite{Martin:1997ns}.
The classical potential of the SM Higgs field $H$ is
\begin{equation}
V = \mu^{2} |H|^2 + \lambda |H|^4.
\end{equation}
Since the SM requires the \textit{vacuum expectation value} (VEV) for $H$, $<H>$, at the minimum of the potential non-vanishing, this only satisfied if $\mu^{2} < 0$ and $\lambda > 0$.
However, the parameter $\mu^{2}$ receives enormous radiative corrections causing it ultraviolet divergent as shown in Equation~\ref{eq:mu2_correction}.
\begin{equation}
\mu^{2} = \mu_{bare}^{2} - \frac{|\lambda_{f}|^{2}}{8\pi}\Lambda_{UV}^{2} + \mathcal{O}(\Lambda_{UV}^{2})
\label{eq:mu2_correction}
\end{equation}
where $\mu_{bare}$ is the Higgs mass, $- \frac{|\lambda_{f}|^{2}}{8\pi}\Lambda_{UV}^{2}$ is the one-loop correction, and $\Lambda_{UV}$ is an ultraviolet momentum cutoff which is valid up to the Plank scale $10^{19}$~{\GeV}.
The electroweak gauge bosons $W^{\pm}$ and $Z^{0}$ obtain their finite masses from $<H>$ so the $\mu^{2}$ cannot be divergent.
There must some unknown mechanism to protect from divergency.

%%%
%%%
%%%

\subsection{Dark matter and dark energy}
\label{subsec:sm_dm}
The matters we know today compose of only 5\%~\cite{arXiv:1212.5225v3,Ade:2013sjv} of the content of the universe and the remaining part is something we don't know.
This unknown matter is called \textit{Dark Matter} (DM)~\cite{j.physrep.2004.08.031} which makes up about 27\% of the universe and the rest 68\% are called \textit{Dark Energy} (DE)~\cite{arXiv:1212.5225v3,Ade:2013sjv}.
Because DM interacts weakly and doesn't interact with the electromagnetic force, it doesn't absorb, emit, or reflect light causing it hard to detect directly. 
The name DM comes from it is invisible.
Dark energy distributes evenly in both space and time throughout the universe so it doesn't dilute when the universe expands.
The observed scientific data hints the presence of DE is necessary to explain the accelerated expansion of the universe.

%%%
%%%
%%%

\subsection{Grand Unification}
\label{subsec:sm_grand_unification}
Maxwell unified the electricity and magnetism into electromagnetism in the 1860s.
About a century later, physicists successfully developed theory of electroweak which links the electromagnetism and the weak force.
Because of the triumph of electroweak theory, theorists raise the question of the possibility to unify all forces.
The \textit{Grand Unified Theory} (GUT)~\cite{0031-9112-37-10-029}, which tries to link three of the four known forces together, is developed in the mid-1970s by theorists
The GUT proposes that the electromagnetic force, weak force, and strong force unify to one force at the GUT scale, $\Lambda_{GUT} \approx 10^{16}$~{\GeV}.
So the three running coupling constants~\cite{RevModPhys.84.1527} are expected to be converged at the GUT scale.
However, the current experiment results show the coupling constants still different as shown in Figure~\ref{fig:sm_coulping_constants}.

\begin{figure}[htbp]
\begin{center}
\includegraphics[scale=0.4]{Ur0ol.pdf}
\caption{The measured running coupling constants in the SM (left) and prediction in the GUT (right).
The three lines show the inverse value of the coupling constant for the three fundamental forces.
This figure is taken from~\cite{Ur0ol}.
}
\label{fig:sm_coulping_constants}
\end{center}
\end{figure}

%%%
%%%
%%%

\subsection{More questions}
\label{subsec:sm_more_questions}
There are some more interesting questions which we don't know the answers.
For example, we don't know the reason why there are 61 elementary particles and more than 20 arbitrary parameters in the SM.
Also, the SM doesn't explain why there are only three generations.
The amount of matter and anti-matter are equal at the beginning of the universe based on the prediction of the SM but the matter dominates in the currently universe which the SM couldn't answer the reason why.

In order to answer these questions, there are many theories being developed on the top of SM but none of them has been observed.
One of the most probable candidate for answering these question is supersymmetry which will be introduced in the next chapter.
