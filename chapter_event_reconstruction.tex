Candidate events are required to have at least a reconstructed $pp$ interaction vertex with at least two $\pt > 400$~{\MeV} associated tracks.
The vertex with the largest $\sum \pt^{2}$ of the associated tracks is selected as the primary vertex of the event. 
In this chapter, the various object reconstructions and  identifications in the ATLAS experiment are presented.
The electron, muon, and tau objects are presented in Sect.~\ref{subsec:event_electrons},~\ref{subsec:event_muons}, and~\ref{subsec:event_Taus}, respectively.
Followed by the photons in Sect.~\ref{subsec:event_photons}, jets in Sect.~\ref{subsec:event_jets}, and \met in Sect.~\ref{subsec:event_met}.
Finally, the signal region selection is described in Sect.~\ref{sec:event_signal_region_selection}.
More information is provided for the electron identification, reconstruction, and isolation in the App.~\ref{app:electron_reconstruction_identification_isolation}.

%%%
%%%
%%%

\section{Object selections}
\label{sec:event_object_selections}
This section presents the object definition and selection in the analysis.
The definition of objects used in this analysis are based on the recommendations by Combined Performance groups and are summarized in Table~\ref{tab:event_object_definitions}.
The objects are divided into two categories: preselected and signal objects where signal objects are a subset of preselected objects.
Unless otherwise stated, the recommendations implemented in \texttt{SUSYTools-00-08-69}~\cite{SUSYToolsV8} and \texttt{AnalysisBase 2.4.37}~\cite{AnalysisBase} are used for all the objects.

\begin{table}[ht]
    \begin{center}
        {\scriptsize
            \begin{tabular}{lll}
                \hline
                \hline
                Property           & Preselected object                     & Signal object\\
                \hline
                \textbf{Electrons} &                                        &\\
                Kinematic          & $\pt > 4.5$~{\GeV}                     & $\pt > 4.5$~{\GeV}, $|\eta| < 2.47$ (include crack)\\
                Identification     & \texttt{VeryLooseLLH}                  & \texttt{TightLLH}\\
                Isolation          & -                                      & \texttt{GradientLoose}\\
                Impact parameter   & $|z_{0} \sin\theta| < 0.5$~mm          & $|d_{0}/\sigma(d_{0})| < 5$, $|z_{0} \sin\theta| < 0.5$~mm\\
                Reco algorithm     & Veto \texttt{author==16}               & Veto \texttt{author==16}\\
                \hline
                \textbf{Muons}     &                                        &\\
                Kinematic          & $\pt > 4$~{\GeV}                       & $\pt > 4$~{\GeV}, $|\eta| < 2.5$\\
                Identification     & \texttt{Medium}                        & \texttt{Medium}\\
                Isolation          & -                                      & \texttt{FixedCutTightTrackOnly}\\
                Impact parameter   & $|z_{0} \sin\theta| < 0.5$~mm          & $|d_{0}/\sigma(d_{0})| < 3$, $|z_{0} \sin\theta| < 0.5$~mm\\
                \hline
                \textbf{Jets}
                Kinematic          & $\pt > 20$~{\GeV}, $|\eta| < 4.5$      & $\pt > 30$~{\GeV}, $|\eta| < 2.8$\\
                Clustering         & Anti-$k_{t}$ $R = 0.4$ \texttt{EMTopo} & Anti-$k_{t}$ $R = 0.4$ \texttt{EMTopo}\\
                Pileup mitigation  & -                                      & JVT \texttt{Medium} for $\pt < 60$~{\GeV}, $|\eta| < 2.4$\\
                $b$-tagging        & -                                      & $\pt > 20$~{\GeV}, $|\eta| < 2.5$, \texttt{MV2c10} \texttt{FixedCutBeff} 85\%\\
                \hline
                \hline
            \end{tabular}
        }
    \end{center}
    \caption{Summary of objec definitions.}
    \label{tab:event_object_definitions}
\end{table}%

%%%
%%%
%%%

\subsection{Electrons}
\label{subsec:event_electrons}
In the ATLAS experiment, electrons are reconstructed using the information from the ID tracks matched to energy clusters in the ECAL.






The preselected electrons used in this analysis have to satisfy $\pt > 4.5$~{\GeV} and $|\eta| < 2.47$ and pass the likelihood-based \texttt{VeryLooseLLH} identification.
The electron tracks are required to satisfy the longitudinal impact parameter $|z_{0}\sin\theta| < 0.5$~mm.
The electrons coming from the photon conversion are rejected by veto the \texttt{author==16}.
The signal electrons have a tighter selection criteria.
Besides all the requirements for the preselected electrons, the signal electrons are also required to pass \texttt{TightLLH} identification, \texttt{GradientLoose} isolation, and the transverse impact parameter $|d_{0}/sigma(d_{0})| < 5$ requirements.
A more detail description about the electron reconstruction, identification, and isolation can be found in the App.~\ref{app:electron_reconstruction_identification_isolation}.

%%%
%%%
%%%

\subsection{Muons}
\label{subsec:event_muons}
Similar to the preselected electrons, the preselected muons are reconstructed by combining the tracks information from ID and the muon spectrometer.
The preselected muons used in this analysis have to satisfy $\pt > 4$~{\GeV} and $|\eta| < 2.5$, pass the \texttt{Medium} identification, and require $|z_{0}\sin\theta| < 0.5$~mm on the longitudinal impact parameter.
A tighter requirement is applied on the signal muons which in addition pass the \texttt{FixedCutTightTrackOnly} isolation together with $|d_{0}/\sigma(d_{0})| < 3$ on the transverse impace parameter.
%%%
%%%
%%%

\subsection{Taus}
\label{subsec:event_Taus}

%%%
%%%
%%%

\subsection{Photons}
\label{subsec:event_photons}

%%%
%%%
%%%

\subsection{Jets}
\label{subsec:event_jets}

%%%
%%%
%%%

\subsubsection{$b$-ets}
\label{subsubsec:event_bjets}

%%%
%%%
%%%

\subsection{Missing transverse energy}
\label{subsec:event_met}

%%%
%%%
%%%

\section{Signal region selection}
\label{sec:event_signal_region_selection}


\begin{table}[ht]
    \begin{center}
        {\scriptsize
            \begin{tabular}{ll}
                \hline
                \hline
                Variable                                                               & Common requirement\\
                \hline
                Number of leptons                                                      & = 2\\
                Lepton charge and flavor                                               & $e^{+}e^{-}$ or $\mu^{+}\mu^{-}$\\
                Leading lepton $\pt^{\ell_{1}}$                                        & $> 5$~{\GeV} for electron and muon\\
                Subleading lepton  $\pt^{\ell_{2}}$                                    & $> 4.5$ (4)~{\GeV} for electron (muon)\\
                $\Delta R_{\ell \ell}$                                                 & $> 0.05$\\
                $m_{\ell \ell}$                                                        & $\in$ [1, 60]~{\GeV} excluding [3.0, 3.2]~{\GeV}\\
                \met                                                                   & $> 200$~{\GeV}\\
                Number of jets                                                         & $\ge 1$\\
                Leading jet \pt                                                        & $> 100$~{\GeV}\\
                $\Delta \phi(j_{1}, \mathbf{p}^{\mathrm{miss}}_{\mathrm{T}})$          & $> 2.0$\\
                min($\Delta \phi($any jet, $\mathbf{p}^{\mathrm{miss}}_{\mathrm{T}}))$ & $> 0.4$\\
                Number of $b$-tagged jets                                              & = 0\\
                $m_{\tau \tau}$                                                        & $< 0$ or $> 160$~{\GeV}\\
                \hline
                                                                                       & Electroweakino SRs\\
                \hline
                $\Delta R_{\ell \ell}$                                                 & $< 2$\\
                $m_{T}^{\ell_{1}}$                                                     & $< 70$~{\GeV}\\
                $\met/H_{\mathrm{T}}^{\mathrm{lep}}$                                   & $>$ max(5, 15 - 2 $\frac{m_{\ell \ell}}{1~{\GeV}}$)\\
                Binned in                                                              & $m_{\ell \ell}$\\ 
                \hline
                \hline
            \end{tabular}
        }
    \end{center}
    \caption{Summary of event selection criteria.
    Signal leptons and signal jets are used when applying all requirements.
    The signal region binning is listed in Table~\ref{tab:event_signal_region_binning}.}
    \label{tab:event_signal_region}
\end{table}%

\begin{table}[ht]
    \begin{center}
        {\scriptsize
            \begin{tabular}{lllllllll}
                \hline
                \hline
                \multicolumn{9}{c}{Electroweakino SRs}\\
                \hline
                Exclusive & SR$ee$-$m_{\ell \ell}$, SR$\mu\mu$-$m_{\ell \ell}$ & [1, 3] & [3.2, 5] & [5, 10] & [10, 20] & [20, 30] & [30, 40] & [40, 60]\\
                Inclusive & SR$\ell \ell$-$m_{\ell \ell}$                      & [1, 3] & [1, 5]   & [1, 10] & [1, 20]  & [1, 30]  & [1, 40]  & [1, 60]\\
                \hline
                \hline
            \end{tabular}
        }
    \end{center}
    \caption{The signal region binnings for the electroweakino SRs.
    The SR is defined by a $m_{\ell \ell}$ range in {\GeV}.
    The exclusive bins are used to set the exclusion limits on the model and the inclusive bins are used to set the model-independent limits.}
    \label{tab:event_signal_region_binning}
\end{table}%
