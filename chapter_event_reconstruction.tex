Candidate events are required to have at least a reconstructed $pp$ interaction vertex with at least two $\pt > 400$~{\MeV} associated tracks.
The vertex with the largest $\sum \pt^{2}$ of the associated tracks is selected as the primary vertex of the event. 
In this chapter, the various object reconstructions and  identifications in the ATLAS experiment are presented.
The electron, muon, and tau objects are presented in Sect.~\ref{subsec:event_electrons},~\ref{subsec:event_muons}, and~\ref{subsec:event_Taus}, respectively.
Followed by the photons in Sect.~\ref{subsec:event_photons}, jets in Sect.~\ref{subsec:event_jets}, and \met in Sect.~\ref{subsec:event_met}.
Finally, the signal region selection is described in Sect.~\ref{sec:event_signal_region_selection}.

%%%
%%%
%%%

\section{Object selections}
\label{sec:event_object_selections}
This section presents the object definition and selection in the analysis.
The general object selections for ATLAS are described followed by the specific selections used for this analysis.
The definition of objects used in this analysis are based on the recommendations by Combined Performance groups and are summarized in Table~\ref{tab:event_object_definitions}.
The objects are divided into two categories: preselected and signal objects where signal objects are a subset of preselected objects.
Unless otherwise stated, the recommendations implemented in \texttt{SUSYTools-00-08-69}~\cite{SUSYToolsV8} and \texttt{AnalysisBase 2.4.37}~\cite{AnalysisBase} are used for all the objects.

\begin{table}[ht]
    \begin{center}
        {\scriptsize
            \begin{tabular}{lll}
                \hline
                \hline
                Property           & Preselected object                     & Signal object\\
                \hline
                \textbf{Electrons} &                                        &\\
                Kinematic          & $\pt > 4.5$~{\GeV}                     & $\pt > 4.5$~{\GeV}, $|\eta| < 2.47$ (include crack)\\
                Identification     & \texttt{VeryLooseLLH}                  & \texttt{TightLLH}\\
                Isolation          & -                                      & \texttt{GradientLoose}\\
                Impact parameter   & $|z_{0} \sin\theta| < 0.5$~mm          & $|d_{0}/\sigma(d_{0})| < 5$, $|z_{0} \sin\theta| < 0.5$~mm\\
                Reco algorithm     & Veto \texttt{author==16}               & Veto \texttt{author==16}\\
                \hline
                \textbf{Muons}     &                                        &\\
                Kinematic          & $\pt > 4$~{\GeV}                       & $\pt > 4$~{\GeV}, $|\eta| < 2.5$\\
                Identification     & \texttt{Medium}                        & \texttt{Medium}\\
                Isolation          & -                                      & \texttt{FixedCutTightTrackOnly}\\
                Impact parameter   & $|z_{0} \sin\theta| < 0.5$~mm          & $|d_{0}/\sigma(d_{0})| < 3$, $|z_{0} \sin\theta| < 0.5$~mm\\
                \hline
                \textbf{Jets}
                Kinematic          & $\pt > 20$~{\GeV}, $|\eta| < 4.5$      & $\pt > 30$~{\GeV}, $|\eta| < 2.8$\\
                Clustering         & Anti-$k_{t}$ $R = 0.4$ \texttt{EMTopo} & Anti-$k_{t}$ $R = 0.4$ \texttt{EMTopo}\\
                Pileup mitigation  & -                                      & JVT \texttt{Medium} for $\pt < 60$~{\GeV}, $|\eta| < 2.4$\\
                $b$-tagging        & -                                      & $\pt > 20$~{\GeV}, $|\eta| < 2.5$, \texttt{MV2c10} \texttt{FixedCutBeff} 85\%\\
                \hline
                \hline
            \end{tabular}
        }
    \end{center}
    \caption{Summary of objec definitions used in this analysis.}
    \label{tab:event_object_definitions}
\end{table}%

%%%
%%%
%%%

\subsection{Electrons}
\label{subsec:event_electrons}

%%%
%%%
%%%

\subsubsection{General electron reconstruction and identification}
\label{subsubsec:event_electrons_general}
In the ATLAS experiment, electron\footnote{Electrons and positrons are collectively referred to as electrons.} objects are reconstructed and identified using the information from the ID tracks matched to energy clusters in the ECAL.
Electron candidates with $\pt > 4$~{\GeV} and $|\eta| < 2.47$ are selected using the tag-and-probe method for $Z \to ee$ and $J/\psi \to ee$ processes.
Three likelihood based electron identification algorithms, \texttt{Loose}, \texttt{Medium}, and \texttt{Tight} are applied to determine the signal-like reconstructed electron candidates.
These three identifications use the same variables to define the likelihood discriminant but with different selection criteria.
Depending on the electron identification used, the reconstruction efficiency varies from 78 to 90\% and increases with \met.
The electron isolation varies between 90\% and 99\% depending on the isolation selection criteria.
More detail about the electron reconstruction performance can be found in Ref.~\cite{ATLAS:2016iqc} and a detail description about the electron isolation, which is my authorship project, can be found in the App.~\ref{app:electron_isolation}.

%%%
%%%
%%%

\subsubsection{Specific to this analysis}
\label{subsubsec:event_electrons_specific}
The preselected electrons used in this analysis have to satisfy $\pt > 4.5$~{\GeV} and $|\eta| < 2.47$ and pass the likelihood-based \texttt{VeryLooseLLH} identification.
The electron tracks are required to satisfy the longitudinal impact parameter $|z_{0}\sin\theta| < 0.5$~mm.
The electrons coming from the photon conversion are rejected by veto the \texttt{author==16}.
The signal electrons have a tighter selection criteria.
Besides all the requirements for the preselected electrons, the signal electrons are also required to pass \texttt{TightLLH} identification, \texttt{GradientLoose} isolation, and the transverse impact parameter $|d_{0}/\sigma(d_{0})| < 5$ requirements.

%%%
%%%
%%%

\subsection{Muons}
\label{subsec:event_muons}

%%%
%%%
%%%

\subsubsection{General muon reconstruction and identification}
\label{subsubsec:event_muons_general}
In the ATLAS experiment, muon objects are reconstructed and identified using the information from ID and muon spectrometer in the $\pt > 4$~{\GeV} and $|\eta| < 2.7$ region.
Muon candidates are identified by applying quality requirements to suppress background which mainly come from pion and kion decays.
Four categories of muon identification, \texttt{Medium}, \texttt{Loose}, \texttt{Tight}, and \texttt{High}-\pt are provided for different physics analyses.
The \texttt{Medium} minimizes the systematic uncertainties and is provided as the default selection for muons in ATLAS.
The \texttt{Loose} maximize the reconstruction efficiency and is used for multilepton final state analyses.
The \texttt{Tight} maximize the purity of muons and the \texttt{High}-\pt maximize the momentum resolution for $\pt > 100$~{\GeV}.
The muon reconstruction efficiency is about 99\% in $5 < \pt < 100$~{\GeV} and $|\eta| < 2.5$ phase space.
The muon isolation efficiency varies between 93\% and 100\% depending on the isolation selection criteria.
More detail about the muon reconstruction performance can be found in Ref.~\cite{Aad:2016jkr}.

%%%
%%%
%%%

\subsubsection{Specific to this analysis}
\label{subsubsec:event_muons_specific}
%Similar to the preselected electrons, the preselected muons are reconstructed by combining the tracks information from ID and the muon spectrometer.
The preselected muons used in this analysis have to satisfy $\pt > 4$~{\GeV} and $|\eta| < 2.5$, pass the \texttt{Medium} identification, and require $|z_{0}\sin\theta| < 0.5$~mm on the longitudinal impact parameter.
A tighter requirement is applied on the signal muons which in addition pass the \texttt{FixedCutTightTrackOnly} isolation together with $|d_{0}/\sigma(d_{0})| < 3$ on the transverse impace parameter.

%%%
%%%
%%%

\subsection{Taus}
\label{subsec:event_Taus}

%%%
%%%
%%%

\subsubsection{General $\tau$ reconstruction and identification}
\label{subsubsec:event_taus_general}
The mass of $\tau$ lepton is 1.77~{\GeV} and the decay length is 80~$\mu$m which is too short to make $\tau$ reaches the active region of the ATLAS detector.
The $\tau$ can decay either leptonically ($\tau \to \ell \nu_{\ell}$, $\ell = e, \mu$) or hadronically ($\tau \to$ hadrons + $\nu_{\tau}$).
The hadronic tau decays are about 65\% of all possible decay modes and the decay products contain one charged pions (22\%) or three charged pions (72\%) of all cases.
Tau candidates are seeded by jets using the method described in Ref.~\cite{ATLAS:2017mpa} and they are required to have $\pt > 10$~{\GeV} and $|\eta| < 2.5$ but veto the candidates in the crack region $1.37 < |\eta| < 1.52$.
A boosted decision tree (BDT) based algorithm is used to identify the $\tau$ candidate and to reject backgrounds from quark- and gluon-initiated jets.
Three identifications, \texttt{Loose}, \texttt{Medium}, and \texttt{Tight} are provided with the efficiency 60\%, 55\%, and 45\% for 1-track and 50\%, 40\%, and 30\% for 3-tracks, respectively.
More detail about the $\tau$ lepton reconstruction and identification performance can be found in Ref.~\cite{ATLAS:2017mpa}.

%%%
%%%
%%%

\subsubsection{Specific to this analysis}
\label{subsubsec:event_taus_specific}

%%%
%%%
%%%

\subsection{Photons}
\label{subsec:event_photons}

%%%
%%%
%%%

\subsubsection{General photons reconstruction and identification}
\label{subsubsec:event_photons_general}
In the ATLAS experiment, photons are reconstructed using the tracking information in ID and the energy deposits in the CAL.
To distinguish prompt photons\footnote{Prompt photons are photons not originating from hadron decays} from background photons, the photon identification is based on a set of rectangular cuts on several discriminating variables computed from the energy deposited in the ECAL and from the shower leakage to the HCAL.
The photon identification is separately applied to the converted and unconverted photons with $25 \le E_{\mathrm{T}} \le 1500$~{\GeV} and 4 $|\eta|$ intervals.
Two identifications, \texttt{Loose} and \texttt{Tight}, are provided.
The \texttt{Loose} provides high efficiency with low jet rejection and the \texttt{Tight}, which is recommanded for the analyses by the Combined Performance groups, provides high fake photon rejection and good efficiency.
The \texttt{Tight} identification efficiency starts from 84\% at low $E_{\mathrm{T}}$ and reaches around 98\% in $1.37 < |\eta| < 1.81$ region for the unconverted photons.
Similar to the unconverted photons, the efficiency for the converted photons increases with energy and reaches up to 98\%.
More detail about the photon reconstruction and identification can be found in Ref.~\cite{ATLAS:2011kuc}.

%%%
%%%
%%%

\subsubsection{Specific to this analysis}
\label{subsubsec:event_photons_specific}
Photons are required to pass \texttt{Tight} identification and have $\pt > 25$~{\GeV}.

%%%
%%%
%%%

\subsection{Jets}
\label{subsec:event_jets}

%%%
%%%
%%%

\subsubsection{General jets reconstruction}
\label{subsubsec:event_jets_general}
In the ATLAS experiment, jets are reconstructed using the anti-$k_{t}$ algorithm with radius parameter $R = 0.4$.
The reconstruction algorithm uses calorimeter topological clusters in $|\eta| < 4.5$ as input.
Four jet cleaning selections, \texttt{Looser}, \texttt{Loose}, \texttt{Medium}, and \texttt{Tight}, are provided to reject the background.
The \texttt{Looser} has the highest efficiency, $\sim$99.8\%, and the \texttt{Tight} has the highest background rejection with efficiency 85\% at $\pt = 25$~{\GeV} and 98\% at $\pt > 50$~{\GeV}.
More detail about the jets reconstruction using anti-$k_{t}$ algorithm can be found in Ref.~\cite{Cacciari:2008gp}.

%%%
%%%
%%%

\subsubsection{$b$-tagging}
\label{subsubsec:event_bjets}
In the ATLAS experiment, it is very important to identify jets containing $b$ hadrons from light flavor jets\footnote{The light flavor jets mean jets containing $u$, $d$, $s$, $c$, or gluons.}.
Many $b$-tagging algorithms were developed to maintain high $b$-tagging efficiency of real $b$-jets and to retain very low misidentification efficiency of the light flavor jets.
The new developed multivariable algorithm, MV2, improving the $c$-jet rejection $\sim$40\% at 77\% $b$-tagging efficiency and the rejection power at high $b$-jet \pt is also improved.
More detail about the $b$-tagging can be found in Ref.~\cite{ATL-PHYS-PUB-2015-022, ATL-PHYS-PUB-2016-012}.

%%%
%%%
%%%

\subsubsection{Specific to this analysis}
\label{subsubsec:event_jets_specific}
The preselected jets are reconstructed with the anti-$k_{t}$ algorithm with radius parameter $R = 0.4$ and required $\pt > 20$~{\GeV} and $|\eta| < 4.5$.
Jets with $\pt < 60$~{\GeV} and $|\eta| < 2.4$ are required to satisfy \texttt{Medium} jet vertex tagger requirement which can suppress pileup jets.
The MV2c10 $b$-tagging algorithm with an 85\% efficiency is applied on the preselected jets with $|\eta| < 2.5$.
The signal jets are required to satisfy $\pt > 20$~{\GeV} and $|\eta| < 2.8$.

%%%
%%%
%%%

\subsection{Missing transverse energy}
\label{subsec:event_met}
The missing transverse energy \met is defined as the negative vector sum of \pt of all reconstructed objects including leptons, jets, and soft term as show in Eq.~(\ref{eq:event_met}).
The soft term is constructed from all tracks associated to the primary vertex but not associated with any physics object.
There two kinds of soft term, calorimeter based soft term (CST) and track based soft term (TST) are used in \met calculation.
The CST \met is constructed form the energy deposits in the CAL not associated with hard objects and the TST \met is built form ID tracks which not match to any reconstructed object.
More detail about the \met reconstruction performance can be found in Ref.~\cite{ATL-PHYS-PUB-2015-023}.
%
\begin{equation}
    \met = - (\mathbf{p}_{\mathrm{T}}^{\mathrm{hard}} + \mathbf{p}_{\mathrm{T}}^{\mathrm{soft}})
    \label{eq:event_met}
\end{equation}
%


%%%
%%%
%%%

\section{Signal region selection}
\label{sec:event_signal_region_selection}


\begin{table}[ht]
    \begin{center}
        {\scriptsize
            \begin{tabular}{ll}
                \hline
                \hline
                Variable                                                               & Common requirement\\
                \hline
                Number of leptons                                                      & = 2\\
                Lepton charge and flavor                                               & $e^{+}e^{-}$ or $\mu^{+}\mu^{-}$\\
                Leading lepton $\pt^{\ell_{1}}$                                        & $> 5$~{\GeV} for electron and muon\\
                Subleading lepton  $\pt^{\ell_{2}}$                                    & $> 4.5$ (4)~{\GeV} for electron (muon)\\
                $\Delta R_{\ell \ell}$                                                 & $> 0.05$\\
                $m_{\ell \ell}$                                                        & $\in$ [1, 60]~{\GeV} excluding [3.0, 3.2]~{\GeV}\\
                \met                                                                   & $> 200$~{\GeV}\\
                Number of jets                                                         & $\ge 1$\\
                Leading jet \pt                                                        & $> 100$~{\GeV}\\
                $\Delta \phi(j_{1}, \mathbf{p}^{\mathrm{miss}}_{\mathrm{T}})$          & $> 2.0$\\
                min($\Delta \phi($any jet, $\mathbf{p}^{\mathrm{miss}}_{\mathrm{T}}))$ & $> 0.4$\\
                Number of $b$-tagged jets                                              & = 0\\
                $m_{\tau \tau}$                                                        & $< 0$ or $> 160$~{\GeV}\\
                \hline
                                                                                       & Electroweakino SRs\\
                \hline
                $\Delta R_{\ell \ell}$                                                 & $< 2$\\
                $m_{T}^{\ell_{1}}$                                                     & $< 70$~{\GeV}\\
                $\met/H_{\mathrm{T}}^{\mathrm{lep}}$                                   & $>$ max(5, 15 - 2 $\frac{m_{\ell \ell}}{1~{\GeV}}$)\\
                Binned in                                                              & $m_{\ell \ell}$\\ 
                \hline
                \hline
            \end{tabular}
        }
    \end{center}
    \caption{Summary of event selection criteria.
    Signal leptons and signal jets are used when applying all requirements.
    The signal region binning is listed in Table~\ref{tab:event_signal_region_binning}.}
    \label{tab:event_signal_region}
\end{table}%

\begin{table}[ht]
    \begin{center}
        {\scriptsize
            \begin{tabular}{lllllllll}
                \hline
                \hline
                \multicolumn{9}{c}{Electroweakino SRs}\\
                \hline
                Exclusive & SR$ee$-$m_{\ell \ell}$, SR$\mu\mu$-$m_{\ell \ell}$ & [1, 3] & [3.2, 5] & [5, 10] & [10, 20] & [20, 30] & [30, 40] & [40, 60]\\
                Inclusive & SR$\ell \ell$-$m_{\ell \ell}$                      & [1, 3] & [1, 5]   & [1, 10] & [1, 20]  & [1, 30]  & [1, 40]  & [1, 60]\\
                \hline
                \hline
            \end{tabular}
        }
    \end{center}
    \caption{The signal region binnings for the electroweakino SRs.
    The SR is defined by a $m_{\ell \ell}$ range in {\GeV}.
    The exclusive bins are used to set the exclusion limits on the model and the inclusive bins are used to set the model-independent limits.}
    \label{tab:event_signal_region_binning}
\end{table}%
