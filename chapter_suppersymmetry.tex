The Standard Model of particle physics (SM)~\cite{BF02726525,0029-55826190469-2,PhysRevLett.19.1264,Herrero:1998eq,CBO9780511791406} gets a stupendous success in expecting and explaining the physics phenomena of the elementary particles.
However, the Standard Model leaves several open questions unanswered as mentioned in Section~\ref{sec:sm_bsm}.
Many different models of new physics were proposed to explain those unanswered questions.
Among these new models, the \textit{supersymmetry} (SUSY)~\cite{Wess:1974tw,Lykken:1996xt,Drees:1996ca,Martin:1997ns,Bilal:2001nv,Argyres:2001eva,Peskin:2008nw,CBO9780511619250,Shadmi:2017qdk} wins most physicists' favour.
The SUSY proposed by Wess and Zumino~\cite{Wess:1974tw} at early 1970s is a symmetry relating bosonic and fermionic degrees of freedom.
It extends the Standard Model by requiring every SM boson/fermion has a fermonic/bosonic supersymmetric partner and vice versa.
The reason why physicists favour SUSY is described in Section~\ref{sec:susy_why_susy}.  
The introduction of the SUSY is given in Section~\ref{sec:susy_intro} and the formulaism is depicted in Section~\ref{sec:susy_}.
The \textit{Radiative Natural SUSY} (RNS) and the \textit{Non-Universal Higgs Mass model} with two extra parameters (NUHM2) are given in Section~\ref{sec:susy_rns} and ~\ref{sec:susy_nuhm2}, respectively.

%%%
%%%
%%%

\section{Why supersymmetry}
\label{sec:susy_why_susy}
The Standard Model leaves several unanswered questions, for example, the hierarchy problem (Section~\ref{subsec:sm_hierarchy_problem}), what is the candidates of dark matter (Section~\ref{subsec:sm_dm}), and why don't the running coupling constants unify at GUT level (Section~\ref{subsec:sm_grand_unification}).
The SUSY provides good explanations for these questions.

%%%
%%%
%%%

\subsubsection{Hierarchy problem}
\label{subsubsec:susy_hierarchy_problem}
The Standard Model expects the Higgs squared mass divergent at the Plank scale $\sim 10^{19}$~{\GeV}.
However, the $W^{\pm}$ and $Z^{0}$ gauge bosons obtained their finite mass through the Higgs mechanism indicates the Higgs squared mass must be finite.
The Figure~\ref{fig:susy_one_loop_corrections} shows the Feymann diagram for the one loop correction to the Higgs squared mass due to a fermion $f$ and a scalar $S$.

\begin{figure}[htbp]
\begin{center}
\includegraphics[scale=0.5]{Figure-1-One-loop-quantum-corrections-to-the-Higgs-squared-mass-parameter-m-2-H-from.png}
\caption{The Feymann diagram for the one loop correction to the Higgs squared mass due to (a) a fermion $f$ and (b) a scalar $S$.
The figure is taken from~\cite{Martin:1997ns}.}
\label{fig:susy_one_loop_corrections}
\end{center}
\end{figure}

The corrections are
%
\begin{align}
\Delta m_{H}^{2} &= - \frac{|\lambda_{f}^{2}|}{8\pi^{2}} \Lambda_{UV}^{2} + \cdots, \quad \mathrm{fermion}\\
\Delta m_{H}^{2} &= \frac{\lambda_{S}}{16\pi^{2}} \Lambda_{UV}^{2} + \cdots, \quad \mathrm{boson}
\end{align}
%
where the $\Lambda_{UV}$ is an ultraviolet momentum cutoff which is valid up to the Plank scale $10^{19}$~{\GeV}.
The corrections diverge when $\Lambda_{UV}$ becoming very large.
Because the contributions from the fermion and scalar loops have opposite sign, the divergence contributions can be canceled out if there is a scalar loop for each fermonic loop.
The SUSY predicts the existence of the bosonic/fermonic sparticles, therefore, if $\lambda_{S} = 2 |\lambda_{f}^{2}|$ then the SUSY maintain the finiteness of the Higgs squared mass in a natural way.

%%%
%%%
%%%

\subsubsection{Dark matter}
\label{subsubsec:susy_dark_matter}
The dark matter (DM) makes up about 27\% of the universe and it might originate from neutral relic from the early universe.
The cosmology observations of the dark matter indicate that the dark mater should be electrically neutral, cold, massive, and it participates only the weak and gravitational interactions.
Therefore, the dark matter candidate should be a new particle which is \textit{weakly interacting massive particle} (WIMP).
The SUSY requires all the sparticles must be produced in pairs and they decay into stable \textit{lightest supersymmetry particles} (LSP) with odd number.
If there are a lot of sparticles produced in the early Universe, they will have to decayed to LSPs and remain until the present day because the LSP is stable.
The LSP is a weakly interacting massive particle.
It doesn't interact electromagnetically so they cannot be scattered by photon and thus dark.
There are 3 kinds of LSP could be the possible dark matter candidate, the lightest \textit{neutralino}, the lightest \textit{sneutrino} and the \textit{gravitino}.

%%%
%%%
%%%

\subsubsection{Grand Unification}
\label{subsubsec:susy_gut}
The Grand Unified Theory (GUT) tries to unify the strong and electroweak interactions.
There will be only one interaction and one coupling constant at the GUT scale ($\approx 10^{16}$~{\GeV}).
However, the current coupling constants for electromagnetic, weak, and strong interactions don't unified at the GUT scale as shown in the left hand side of Figure~\ref{fig:sm_coulping_constants}.
This problem can be solved by introducing the SUSY which modifies the renormalization group equations and make the running gauge couplings converged at the GUT scale.
The right hand side of Figure~\ref{fig:sm_coulping_constants} shows the running gauge couplings in SUSY.

%%%
%%%
%%%

\section{Introduction of the supersymmetry}
\label{sec:susy_intro}
An brief overview of the SUSY are introduced in this section.
Firstly, the mathematical foundation of the SUSY, superalgebra, is described in Section~\ref{subsec:susy_superalgebra} followed by the superspace and superfields in Section~\ref{subsec:susy_superspace_and_superfields}.

%%%
%%%
%%%

\subsection{Superalgebra}
\label{subsec:susy_superalgebra}
The SUSY is based on the superalgebra which is an extension of space-time Poincar\'{e} algebra.
The Poincar\'{e} group is a product of the Lorentz group and the group of translations in space-time.
A Lorentz group must satisfies the commutation relations
%
\begin{equation}
[J^{+}_{i}, J^{+}_{j}] = i \epsilon_{ijk} J^{+}_{k}, \quad 
[J^{-}_{i}, J^{-}_{j}] = i \epsilon_{ijk} J^{-}_{k}, \quad 
[J^{+}_{i}, J^{-}_{j}] = 0
\label{eq:susy_Lorentz_commutation_relations}
\end{equation}
%
where $i, j, k = 1, 2, 3$.
If the six Lorentz group generators are combined into an antisymmetric second rank tensor generator $M_{\mu\nu}$ where $M_{ij} = \epsilon_{ijk}J_{k}$ and $M_{0i} = -M_{i0} = -K_{i}$\footnote{The $J_{i}$ and $K_{i}$ with $i=1,2,3$ are rotation and boost generators in 3-dimension respectively.} and the generator of the translation groups is $P_{\mu}$, the energy-momentum operator, then the commutation relations of the Poincar\'{e} group are
%
\begin{align}
[P_{\mu}, P_{\nu}] &= 0 ,\\
[M_{\mu \nu}, P_{\lambda}] &= i (g_{\nu \lambda} P_{\mu} - g_{\mu \lambda} P_{\nu}) ,\\
[M_{\mu \nu}, M_{\rho \sigma}] &= -i (g_{\mu \rho} M_{\nu \sigma} - g_{\mu \sigma} M_{\nu \rho} - g_{\nu \rho} M_{\mu \sigma} + g_{\nu \sigma} M_{\mu \rho}) .
\label{eq:susy_Poincare_commutation_relations}
\end{align}
where the metric is 
\begin{equation}
g_{\mu \nu} =
\left(
\begin{array}{cccc}
1 & 0 & 0 & 0\\
0 & -1 & 0 & 0\\
0 & 0 & -1 & 0\\
0 & 0 & 0 & -1   
\end{array}
\right)
\label{eq:susy_metric}
\end{equation}
%
A general spin 1/2 particle state, $\chi$, can be expressed as a \textit{spinor} in the SUSY using two-component spin up $\chi_{+}$ and spin down $\chi_{-}$ column matrices
%
\begin{equation}
\chi = c_{+} \chi_{+} + c_{-} \chi_{-}
= c_{+} \left(\begin{matrix}1\\0\end{matrix}\right) + c_{-} \left(\begin{matrix}0\\1\end{matrix}\right)
= \left(\begin{matrix}c_{+}\\c_{-}\end{matrix}\right)
\label{eq:susy_spinor}
\end{equation}
%
The solution of the Dirac equation\footnote{The Dirac equation is $(i \gamma^{\mu} \partial_{\mu} - m)\psi = 0$.}, $\psi_{D}\footnote{The Dirac spinor, $\psi_{D}$, is a four-component field which can be expressed using a four-component matrix.}$, can be expressed using the left-handed and right-handed \textit{Weyl spinors} $\psi_{L}$ and $\psi_{R}$ 
%
\begin{equation}
\psi_{D} = \left(\begin{matrix}\psi_{1}\\\psi_{2}\\\psi_{3}\\\psi_{4}\end{matrix}\right)
= \left(\begin{matrix} \left(\begin{matrix}\psi_{1}\\\psi_{2}\end{matrix}\right) \\ \left(\begin{matrix}\psi_{3}\\\psi_{4}\end{matrix}\right) \end{matrix}\right)
= \left(\begin{matrix}\psi_{L}\\\psi_{R}\end{matrix}\right)
\label{eq:susy_Dirac_spinor}
\end{equation}
%

%%%
%%%
%%%

\subsection{Superspace and superfields}
\label{subsec:susy_superspace_and_superfields}

%%%
%%%
%%%

\subsection{$R$-parity}
\label{subsec:susy_r_parity}
The baryon number $B$ and lepton number $L$ are conserved in the Standard Model but violated in the SUSY.
Therefore, a new symmetry called $R$-parity is introduced to eliminate the $B$ and $L$ violating term.
The $R$-parity is defined as
%
\begin{equation}
R \equiv (-1)^{3(B-L)+2s}
\label{eq:susy_r_parity}
\end{equation}
%
where $s$ is the spin of the particle.
All of the SM particles have even $R$-parity ($R$ = +1), while all of the sparticles have odd $R$-parity ($R$ =  1). 
If the $R$-parity is conserved, SUSY predicts that sparticles are produced in pairs in collider experiments.

%%%
%%%
%%%

\subsection{Soft suppersymmetry breaking}
%%%
%%%
%%%

\subsection{The Minimal Supersymmetry Standard Model}
\label{subsec:susy_mssm}
The Minimal Supersymmetry Standard Model (MSSM) is the minimal extension of the Standard Model.
The MSSM contains only the smallest number of superfields and interactions such that the Standard Model particles can keep their current forms.

%%%
%%%
%%%

\subsubsection{Particle content}
\label{subsubsec:susy_particle_content}
%The anticommuting SUSY generator, $Q$, changes the bosons/fermions into fermions/bosons
%\begin{equation}
%Q |\mathrm{boson}\rangle = |\mathrm{fermion}\rangle, \quad Q|\mathrm{fermion}\rangle = |\mathrm{boson}\rangle
%\label{eq:susy_generator}
%\end{equation}
All the super particles, \textit{\textbf{s}particles}\footnote{The super particles of the SM fermions are prefix a "\textit{\textbf{s}}" and the super particles of the SM bosons are suffix an "\textit{\textbf{ino}}". A tilde is added on the symbol of the SM particle to denote its super partner.}, have exactly the same quantum number as their SM particles except the spins differ by $\frac{1}{2}$.
The super partners of the leptons and quarks are called \textit{\textbf{s}leptons} and \textit{\textbf{s}quarks}.
The sleptons and squarks are scalar particles with spin $s=0$ and the left-handed and right-handed states are treated as different particles such that the SM particles and the SUSY \textbf{s}particles have the same number of degree of freedom.
The super partners of gluons are \textit{glu\textbf{ino}s} and there are eight gluinos with spin $s=\frac{1}{2}$. 
The super partners of the gauge bosons $W^{\pm}, Z^{0}$, and $\gamma$, are \textit{gaug\textbf{ino}s}.
The gauginos have spin $s = \frac{1}{2}$.
The super partners of the Higgs bosons\footnote{The Higgs sector contains two charged states $H^{\pm}$ and three neutral states $h^{0}, H^{0}$, and $A^{0}$. The $h^{0} and H^{0}$ are $CP$ even state and $A^{0}$ is $CP$ odd state.} are \textit{Higgs\textbf{ino}s}.
The Higgsinos and gauginos mixing states are two \textit{charginos} $\tilde{\chi}_{1}^{\pm}, \tilde{\chi}_{2}^{\pm}$ and four \textit{neutralinos} $\tilde{\chi}_{1}^{0}, \tilde{\chi}_{2}^{0}, \tilde{\chi}_{3}^{0}, \tilde{\chi}_{4}^{0}$ with spin $s=\frac{1}{2}$.
Table~\ref{tab:susy_particle_contents} shows the particle contents in the MSSM.

\begin{table}[htp]
%\begin{center}
\resizebox{\textwidth}{!}{% <------ Don't forget this %
\begin{tabular}{ccccccc}
\hline
\hline
Supermultiplet & Names & Symbol & spin 0 & spin 1/2 & spin 1 & $SU(3)_{C} \otimes SU(2)_{L} \otimes U(1)_{Y}$\\
\hline
\multirow{3}{*}{Chiral} & \multirow{3}{3cm}{squarks, quarks ($\times 3$ families)} &Q & ($\widetilde{u}_{L}$, \quad $\widetilde{d}_{L})$ & $(u_{L}, \quad d_{L})$ & - & $\mathbf{3} \otimes \mathbf{2}\otimes \frac{1}{6}$\\
& & $\overline{u}$ & $\widetilde{u}^{*}_{R}$ & $u^{\dag}_{R}$ & - & $\overline{\mathbf{3}} \otimes \mathbf{1} \otimes -\frac{2}{3}$\\
& & $\overline{d}$ & $\widetilde{d}^{*}_{R}$ & $d^{\dag}_{R}$ & - & $\overline{\mathbf{3}} \otimes \mathbf{1} \otimes \frac{1}{3}$\\
\hline
\multirow{2}{*}{Chiral} & \multirow{2}{3cm}{sleptons, leptons ($\times 3$ families)} & L & $(\widetilde{\nu}, \quad \widetilde{e}_{L})$ & $(\nu, \quad e_{L})$ & - & $\mathbf{1} \otimes \mathbf{2} \otimes -\frac{1}{2}$\\
& & $\overline{e}$ & $\widetilde{e}^{*}_{R}$ & $e^{\dag}_{R}$ & - & $\mathbf{1} \otimes \mathbf{1} \otimes 1$\\
\hline
\multirow{2}{*}{Chiral}  & \multirow{2}{*}{Higgs, higgsinos} & $H_{u}$ & $(H^{+}_{u}, \quad H^{0}_{u})$ & $(\widetilde{H}^{+}_{u}, \quad \widetilde{H}^{0}_{u})$ & - & $\mathbf{1} \otimes \mathbf{2} \otimes +\frac{1}{2}$\\
& & $H_{d}$ & $(H^{0}_{d}, \quad H^{-}_{d})$ & $(\widetilde{H}^{0}_{d}, \quad \widetilde{H}^{-}_{d})$ & - & $\mathbf{1} \otimes \mathbf{2} \otimes -\frac{1}{2}$\\
\hline
\hline
\multirow{3}{*}{Gauge} & gluino, gluon & - & - & $\widetilde{g}$ & $g$ & $\mathbf{8} \otimes \mathbf{1} \otimes 0$\\
& winos, $W$ bosons & - & - & $\widetilde{W}^{\pm}$, $\widetilde{W}^{0}$ & $W^{\pm}$, $W^{0}$ & $\mathbf{1} \otimes \mathbf{3} \otimes 0$\\
& bino, $B$ boson & - & - & $\widetilde{B}^{0}$ & $B^{0}$ & $\mathbf{1} \otimes \mathbf{1} \otimes 0$\\
\hline
\hline
\end{tabular}
%\end{center}
}
\caption{Chiral supermultiplets and gauge supermultiplets in the MSSM.
In the chiral supermultiplets, the spin 0 fields are complex scalars and the spin 1/2 fields are left-handed two-component Weyl spinors.}
\label{tab:susy_particle_contents}
\end{table}%


 

%%%
%%%
%%%

\section{}
\label{sec:susy_}

%%%
%%%
%%%

\section{The radiative natural SUSY}
\label{sec:susy_rns}

%%%
%%%
%%%

\section{The non-universal Higgs model with two extra parameters}
\label{sec:susy_nuhm2}