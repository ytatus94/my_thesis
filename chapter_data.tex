This chapter describes the collision data and simulated event samples used in searching for electroweak production of SUSY states in the compressed scenarios.
The collision data are presented in Sect.~\ref{sec:data_collision_data} and the Monte Carlo (MC) simulated event samples are detailed in Sect.~\ref{sec:data_MC_samples}.
The samples used for searching the strongly-produced SUSY particles in final states with two same-sign or three lepton and jets (SS/3L+jets) can be found in the App.~\ref{sec:app_samples_strong}.


\section{Collision data}
\label{sec:data_collision_data}
The $pp$ collision data used in this analysis were collected by the ATLAS detector at $\sqrt{s} = 13$~{\TeV} during 2015 and 2016 at LHC.
The data corresponds to an integrated luminosity of 36.1 \ifb (3.2 \ifb in 2015 and 32.9 \ifb in 2016) with a combined uncertainty of 2.1\% after applying beam, detector, and data-quality requirements.
The combined uncertainty is derived following the methodology similar to those described in Ref.~\cite{Aaboud:2016hhf}.
The average number of $pp$ interactions per bunch crossing (pile-up) is 13.5 in the 2015 data set and is 25 in the 2016 data set.
The data samples are required to satisfy the following good runs list (GRLs) as recommended by the ATLAS collaboration
%
\begin{itemize}
    \item {\scriptsize \texttt{data15\_13TeV.periodAllYear\_DetStatus-v79-repro20-02\_DQDefects-00-02-02\_PHYS\_StandardGRL\_All\_Good\_25ns.xml}}
    \item {\scriptsize \texttt{data16\_13TeV.periodAllYear\_DetStatus-v88-pro20-21\_DQDefects-00-02-04\_PHYS\_StandardGRL\_All\_Good\_25ns.xml}}
\end{itemize}
%
Events are selected using different inclusive \met triggers depending on the run period as listed in Table.~\ref{tab:data_triggers}.
Two new triggers, \texttt{HLT\_mu4\_j125\_xe90\_mht} and \texttt{HLT\_2mu4\_j85\_xe50\_mht}, are developed for compressed scenarios starting from run number 308084.
However, these new triggers only contribute a small gain compared to the inclusive \met triggers.
This analysis uses inclusive \met triggers only.

\begin{table}[htp]
    \begin{center}
        \begin{tabular}{cc}
            \hline
            \hline
            Run period & \met trigger\\
            \hline
            2015 & \texttt{HLT\_xe70\_mht}\\
            A-D3 & \texttt{HLT\_xe90\_mht\_L1XE50}\\
            D4-F1 & \texttt{HLT\_xe100\_mht\_L1XE50}\\
            F1- & \texttt{HLT\_xe110\_mht\_L1XE50}\\
            \hline
            \hline
        \end{tabular}
    \end{center}
    \caption{The inclusive \met trigger used in this analysis.}
    \label{tab:data_triggers}
\end{table}%

%%%
%%%
%%%

\section{Monte Carlo simulated event samples}
\label{sec:data_MC_samples}
MC samples are used to model the SUSY signals and to estimated the SM background.
All SM background MC samples were processed through a detailed ATLAS detector simulation based on {\GEANT}4~\cite{Agostinelli:2002hh} and the SUSY signal samples were simulated by a fast simulation (AF2) that parameterizes the calorimeter response~\cite{ATLAS:2010bfa}.
To simulate the effects of additional $pp$ collisions (pile-up) in the same and nearby bunch crossings, inelastic interactions were generated using the soft QCD processes of {\PYTHIA} 8.186~\cite{Sjostrand:2007gs} with A2 tune~\cite{ATLAS:2012uec} and the MSTW2008LO PDF set~\cite{Martin:2009iq}.
These MC events were overlaid onto each simulated hard-scatter event and reweighted to match the pile-up conditions observed in the data.

%%%
%%%
%%%

\subsection{The SM background samples}
\label{subsec:data_sm_bkg_samples}
The {\SHERPA} 2.1.1, 2.2.1, and 2.2.2 ~\cite{Gleisberg:2008ta} were used to produce the $Z^{(*)}/\gamma^{*}$ + jets, diboson, and triboson events.
The matrix elements (ME) were calculated for up to two partons at next-to-leading order (NLO) and up to four partons at leading order (LO) depending on the process.
The $Z^{(*)}/\gamma^{*}$ + jets and diboson samples cover the dilepton invarant masses from 0.5~{\GeV} for $Z^{(*)}/\gamma^{*} \to e^{+}e^{-}/\mu^{+}\mu^{-}$ and from 3.8~{\GeV} for $Z^{(*)}/\gamma^{*} \to \tau^{+}\tau^{-}$.
The {\POWHEG}-Box v1 and v2 interfaced to {\PYTHIA} 6.428 were used to simulated \ttbar and single-top production at NLO in the ME.
The Higgs boson production were generated using {\POWHEG}-Box v2 interfaced to {\PYTHIA} 8.186.
A Higgs boson in association with a $W$ or $Z$ boson production was simulated using MG5\_{\scriptsize A}MC@NLO 2.2.2 with {\PYTHIA} 8.186 and the ATLAS A14 tune.
The processes containing \ttbar and at least one electroweak bosons were produced using MG5\_{\scriptsize A}MC@NLO 2.2.1, 2.2.2, 2.3.2, 2.3.3 with {\PYTHIA} 6.4.28 or 8.186.
These processes were generated at NLO in the ME except for $t + Z$ and $t +$ \ttbar which were produced at LO.
Table~\ref{tab:data_mc_samples} summarizes the event generator configurations of the ME, parton shower, PDF set, and the cross-section normalization.
Except those produced by {\SHERPA} event generator, the \textsc{EVTGEN}\xspace v1.2.0~\cite{Lange:2001uf} was used to model the decay of bottom and charm hadrons in all MC samples.

\begin{table}[ht]
    %\begin{center}
    \resizebox{\textwidth}{!}{% <------ Don't forget this %
        \begin{tabular}{lllll}
            \hline
            \hline
            Process                     & Matrix element                                      & Parton shower   & PDF set         & Cross-section\\
            \hline
            $Z^{(*)}/\gamma^{*}$ + jets & \multicolumn{2}{c}{{\SHERPA} 2.2.1}                                   & NNPDF 3.0 NNLO  & NNLO\\
            Diboson                     & \multicolumn{2}{c}{{\SHERPA} 2.1.1 / 2.2.1 / 2.2.2}                   & NNPDF 3.0 NNLO  & Generator NLO\\
            Triboson                    & \multicolumn{2}{c}{{\SHERPA} 2.2.1}                                   & NNPDF 3.0 NNLO  & Generator LO, NLO\\
            \hline
            \ttbar                      & {\POWHEG}-Box v2                                    & {\PYTHIA} 6.428 & NLO CT10        & NNLO + NNLL\\
            $t$ ($s$-channel)           & {\POWHEG}-Box v1                                    & {\PYTHIA} 6.428 & NLO CT10        & NNLO + NNLL\\
            $t$ ($t$-channel)           & {\POWHEG}-Box v1                                    & {\PYTHIA} 6.428 & NLO CT10f4      & NNLO + NNLL\\
            $t + W$                     & {\POWHEG}-Box v1                                    & {\PYTHIA} 6.428 & NLO CT10        & NNLO + NNLL\\
            \hline
            $h(\to \ell\ell, WW)$       & {\POWHEG}-Box v2                                    & {\PYTHIA} 8.186 & NLO CTEQ6L1     & NLO\\
            $h + W/Z$                   & MG5\_{\scriptsize A}MC@NLO 2.2.2                    & {\PYTHIA} 8.186 & NNPDF 2.3 LO    & NLO\\
            \hline
            \ttbar + $W/Z/\gamma^{*}$   & MG5\_{\scriptsize A}MC@NLO 2.3.3                    & {\PYTHIA} 8.186 & NNPDF 3.0 LO    & NLO\\
            \ttbar + $WW$/\ttbar        & MG5\_{\scriptsize A}MC@NLO 2.2.2                    & {\PYTHIA} 8.186 & NNPDF 2.3 LO    & NLO\\
            $t + Z$                     & MG5\_{\scriptsize A}MC@NLO 2.2.1                    & {\PYTHIA} 6.428 & NNPDF 2.3 LO    & LO\\
            $t + WZ$                    & MG5\_{\scriptsize A}MC@NLO 2.3.2                    & {\PYTHIA} 8.186 & NNPDF 2.3 LO    & NLO\\
            $t$ + \ttbar                & MG5\_{\scriptsize A}MC@NLO 2.2.2                    & {\PYTHIA} 8.186 & NNPDF 2.3 LO    & LO\\
            \hline
            \hline
        \end{tabular}
    }
    %\end{center}
    \caption{The MC simulated samples of SM background process.}
    \label{tab:data_mc_samples}
\end{table}%

%%%
%%%
%%%

\subsection{The SUSY signal samples}
\label{subsec:data_susy_signal_samples}
The NUHM2 model allows the masses of the Higgs doublets $m_{H_{u}}$ and $m_{H_{d}}$ differ from the universal scalar mass $m_{0}$ at the GUT scale.
The parameters of the NUHM2 model were fixed to $m_{0} = 5$~{\TeV}, $m_{A} = 1$~{\TeV}, $A_{0} = -1.6 m_{0}$, $\tan\beta = 15$, and $\mu = 150$~{\GeV}.
The parameter $m_{1/2}$ is a free parameter and it is varied from 350 to 800~{\GeV}.
These parameters are choosen based on Ref.~\cite{Baer:2013xua}.
These parameter settings lead to RNS with low EWFT which keeps the Higgs boson mass about 125~{\GeV}, the masses of $\tilde{g}$ and $\tilde{q}$ about {\TeV} scale, and the light Higgsino mass about $\mu$.
The mass spectra and decay branching ratios were calculated using \textsc{IsAJET}\xspace v7.84~\cite{Baer:1999sp}.
