%% Eduardo Martinez Pedroza.
%%
%% Note from the author:  My dissertation document was created with this file. It passed the general formatting guidelines for PhD dissertations
%% of the Graduate College at the University of Oklahoma. I accept no responsibility in connection with the use of this file. Use it under your own risk.
%%

\documentclass[12pt]{report}

%%%%%%%%%%%%%%%%%%%%%%%%%%%%%%%%%%%%%%%%%%%%%%%%%%%%%%%%%%%%%%%%%%%%%%%%%%%%
%% Standard AMS packages
\usepackage{amssymb, amsmath, amsthm}

%%%%%%%%%%%%%%%%%%%%%%%%%%%%%%%%%%%%%%%%%%%%%%%%%%%%%%%%%%%%%%%%%%%%%%%%%%%%
%%  For mathematical diagrams
\usepackage{xypic}

%%%%%%%%%%%%%%%%%%%%%%%%%%%%%%%%%%%%%%%%%%%%%%%%%%%%%%%%%%%%%%%%%%%%%%%%%%%%
%% For inserting graphic files
\usepackage{graphicx} %

%%%%%%%%%%%%%%%%%%%%%%%%%%%%%%%%%%%%%%%%%%%%%%%%%%%%%%%%%%%%%%%%%%%%%%%%%%%%
%% Hyper-references
%\usepackage[pdftex, plainpages=false]{hyperref}   %To be use with PDF-TEX

%%%%%%%%%%%%%%%%%%%%%%%%%%%%%%%%%%%%%%%%%%%%%%%%%%%%%%%%%%%%%%%%%%%%%%%%%%%%
%% Graduate college required margins
\usepackage[left=1.5in, top=1.3in, right=1.3in, bottom=1.2in]{geometry}

%%%%%%%%%%%%%%%%%%%%%%%%%%%%%%%%%%%%%%%%%%%%%%%%%%%%%%%%%%%%%%%%%%%%%%%%%%%%
%% line spacing
\usepackage{setspace}
    %\singlespacing
    %\onehalfspacing
    \doublespacing


%%%%%%%%%%%%%%%%%%%%%%%%%%%%%%%%%%%%%%%%%%%%%%%%%%%%%%%%%%%%%%%%%%%%%%%%%%%%
%% Personal Environments and Commands
\newtheorem{theorem}{Theorem}[chapter]
\newtheorem{lemma}[theorem]{Lemma}
\newtheorem{proposition}[theorem]{Proposition}
\newtheorem{corollary}[theorem]{Corollary}
\theoremstyle{definition}
\newtheorem{definition}[theorem]{Definition}
\theoremstyle{remark}
\newtheorem{remark}[theorem]{Remark}
\newtheorem{example}{Example}
\numberwithin{equation}{chapter}
\numberwithin{figure}{chapter}

%%%%%%%%%%%%%%%%%%%%%%%%%%%%%%%%%%%%%%%%%%%%%%%%%%%%%%%%%%%%%%%%%%%%%%%%%%%%
%% My commands
\newcommand{\reals}{\mathbb{R}}   %Real numbers' symbol
\newcommand{\hyperbolic}{\mathbb{H}}  %Hyperbolic symbol
\newcommand{\complex}{\mathbb{C}} %Complex numbers' symbol
\newcommand{\integers}{\mathbb{Z}}   %Integer numbers' symbol
\newcommand{\presentation}[2]{\langle \, #1  \, | \, #2 \,  \rangle}
\newcommand{\GammaH}{Cayley (G, X\cup \widetilde{\mathcal{H}}) }
\newcommand{\Hs}{\{ H_i \}_{i=1}^m}


%%%%%%%%%%%%%%%%%%%%%%%%%%%%%%%%%%%%%%%%%%%%%%%%%%%%%%%%%%%%%%%%%%%%%%%%%%%%
%% THESIS %%%%%%%%%%%%%%%%%%%%%%%%%%%%%%%%%%%%%%%%%%%%%%%%%%%%%%%%%%%%%%%%%%%%%%%%%%
%%%%%%%%%%%%%%%%%%%%%%%%%%%%%%%%%%%%%%%%%%%%%%%%%%%%%%%%%%%%%%%%%%%%%%%%%%%%

\begin{document}

\title{Your Thesis title goes here}
\author{Your Full Name}
\date{month year}

\pagenumbering{roman}
{%% PAGE NUMBERING ROMAN

%%%%%%%%%%%%%%%%%%%%%%%%%%%%%%%%%%%%%%%%%%%%%%%%%%%%%%%%%%%%%%%%%%%%%%%%%%%%
%% TITLE PAGE  %%%%%%%%%%%%%%%%%%%%%%%%%%%%%%%%%%%%%%%%%%%%%%%%%%%%%%%%%%%%%%%%%%%%%%%%%%
%%%%%%%%%%%%%%%%%%%%%%%%%%%%%%%%%%%%%%%%%%%%%%%%%%%%%%%%%%%%%%%%%%%%%%%%%%%%
{\singlespacing

\newpage
\thispagestyle{empty}
\begin{center}
{ %\large
UNIVERSITY OF OKLAHOMA
\par
\vspace{0.16in}
GRADUATE COLLEGE
\par
\vspace{1.2in}
%%%%%%%%%%%%%%%%%%%%%%%%%%%%%%5
COMBINATION OF QUASICONVEX SUBGROUPS IN
\par
\vspace{0.17in}
RELATIVELY HYPERBOLIC GROUPS
\par
\vspace{1.2in}
%%%%%%%%%%%%%%%%%%%%%%%%%%%%%%
A DISSERTATION
\par
\vspace{0.17in}
SUBMITTED TO THE GRADUATE FACULTY
\par
\vspace{0.17in}
in partial fulfillment of the requirements for the
\par
\vspace{0.17in}
Degree of
\par
\vspace{0.17in}
DOCTOR OF PHILOSOPHY
\par
\vfill
%%%%%%%%%%%%%%%%%%%%%%%%%%%%%
By
\par
\vspace{0.17in}
EDUARDO MARTINEZ PEDROZA
\par
%\vspace{0.05in}
Norman, Oklahoma
\par
%\vspace{0.05in}
2008
}
\end{center}


%%%%%%%%%%%%%%%%%%%%%%%%%%%%%%%%%%%%%%%%%%%%%%%%%%%%%%%%%%%%%%%%%%%%%%%%%%%%
%% SIGNATURE PAGE  %%%%%%%%%%%%%%%%%%%%%%%%%%%%%%%%%%%%%%%%%%%%%%%%%%%%%%%%%%%%%%%%%%%%%%%%%%
%%%%%%%%%%%%%%%%%%%%%%%%%%%%%%%%%%%%%%%%%%%%%%%%%%%%%%%%%%%%%%%%%%%%%%%%%%%%
\newpage
\thispagestyle{empty}
\begin{center}
{%\large
COMBINATION OF QUASICONVEX SUBGROUPS IN
\par
%\vspace{0.17in}
RELATIVELY HYPERBOLIC GROUPS
\par
\vspace{0.9in}
A DISSERTATION APPROVED FOR THE
\par
%\vspace{0.17in}
DEPARTMENT OF MATHEMATICS
\par
\vspace{0.9in}
BY
\par
\vfill
\begin{flushright}
\begin{tabular}{cr}
\hline
\  \  \  \  \  \  \  & Dr. Noel Brady, Chair\\
 \\ &  \\
  \\ &  \\
\hline
  & Dr. Max Forester \\
\\ & \\
 \\ &  \\
\hline
 & Dr. Darryl McCullough \\
\\   & \\
 \\ &  \\
\hline
 & Dr. Krishnan Shankar \\
 \\ &  \\
  \\ &  \\
\hline
   & Dr. Stephen Weldon \\
\end{tabular}
\end{flushright}
}
\end{center}


%%%%%%%%%%%%%%%%%%%%%%%%%%%%%%%%%%%%%%%%%%%%%%%%%%%%%%%%%%%%%%%%%%%%%%%%%%%%
%% COPYRIGHT PAGE  %%%%%%%%%%%%%%%%%%%%%%%%%%%%%%%%%%%%%%%%%%%%%%%%%%%%%%%%%%%%%%%%%%%%%%%%%%
%%%%%%%%%%%%%%%%%%%%%%%%%%%%%%%%%%%%%%%%%%%%%%%%%%%%%%%%%%%%%%%%%%%%%%%%%%%%
\newpage
\thispagestyle{empty}
\   \
\par
\vfill
\begin{center}
\copyright \    Copyright by EDUARDO MARTINEZ PEDROZA \   2008

All rights reserved.
\end{center}
}



%%%%%%%%%%%%%%%%%%%%%%%%%%%%%%%%%%%%%%%%%%%%%%%%%%%%%%%%%%%%%%%%%%%%%%%%%%%%
%% ACKNOWLEDGEMENTS PAGE  %%%%%%%%%%%%%%%%%%%%%%%%%%%%%%%%%%%%%%%%%%%%%%%%%%%%%%%%%%%%%%%%%%%%%%%%%%
%%%%%%%%%%%%%%%%%%%%%%%%%%%%%%%%%%%%%%%%%%%%%%%%%%%%%%%%%%%%%%%%%%%%%%%%%%%%
\chapter*{Acknowledgements}

First, I wish to express my gratitude to my research advisor, Professor Noel Brady, for teaching me a great deal of mathematics,
for encouraging to explore and work in mathematics,  for his guidance and unconditional support during all these years of graduate school,
and for being a excellent mentor.  I am grateful for having had the opportunity to work with him.

I also wish to thank ....

%%%%%%%%%%%%%%%%%%%%%%%%%%%%%%%%%%%%%%%%%%%%%%%%%%%%%%%%%%%%%%%%%%%%%%%%%%%%
%% TABLE OF CONTENTS PAGE  %%%%%%%%%%%%%%%%%%%%%%%%%%%%%%%%%%%%%%%%%%%%%%%%%%%%%%%%%%%%%%%%%%%%%%%%%%
%%%%%%%%%%%%%%%%%%%%%%%%%%%%%%%%%%%%%%%%%%%%%%%%%%%%%%%%%%%%%%%%%%%%%%%%%%%%

{\singlespacing
\tableofcontents
%\listoffigures
}
\newpage

} %% END PAGE NUMBERING ROMAN


%%%%%%%%%%%%%%%%%%%%%%%%%%%%%%%%%%%%%%%%%%%%%%%%%%%%%%%%%%%%%%%%%%%%%%%%%%%%
%%%%%%%%%%%%%%%%%%%%%%%%%%%%%%%%%%%%%%%%%%%%%%%%%%%%%%%%%%%%%%%%%%%%%%%%%%%%

\pagenumbering{arabic}

%%%%%%%%%%%%%%%%%%%%%%%%%%%%%%%%%%%%%%%%%%%%%%%%%%%%%%%%%%%%%%%%%%%%%%%%%%%%
%% INTRODUCTION %%%%%%%%%%%%%%%%%%%%%%%%%%%%%%%%%%%%%%%%%%%%%%%%%%%%%%%%%%%%%%%%%%
%%%%%%%%%%%%%%%%%%%%%%%%%%%%%%%%%%%%%%%%%%%%%%%%%%%%%%%%%%%%%%%%%%%%%%%%%%%%
\chapter{Introduction}

In this thesis, we consider problems concerning .....

If $G$ is a countable group and $\mathcal{H}$ is a collection of subgroups of $G$, the notion of \emph{relative hyperbolicity} for the pair $(G, \mathcal{H})$  has been defined by
different authors ~\cite{BO99, DS05, Fa98, Gr87, GM06, HK08, Os06, Ya99}.  All these definitions are equivalent when ....

\begin{displaymath}
\xymatrix{
                                                                            & & L \ar@{->}[lldd]_{\sigma} \ar@{->}[rrdd]^{\tau}         & &                                                                  \\
                                                                            & &                                                                                     & &                                                                      \\
    H \ar@{->}[rr]^{\imath_H} \ar@{->}[dddrr]_\alpha                &   &   H\ast_L K \ar@{-->}[ddd]^{\exists ! \rho}   & &  K \ar@{->}[ll]_{\imath_K}  \ar@{->}[dddll]^\beta           \\
                                                                            & &                                                                                     & &                                                                         \\
                                                                            & &                                                                                     & &                                                                         \\
                                                                            & & G                                                                                   & &
}
\end{displaymath}


\section{Main Results and Applications}

The main results are .....

\subsection{Examples of Relatively Hyperbolic Groups}

\begin{theorem}[S.M.\ Gersten \cite{Ge96} - B.\ Bowditch \cite{BO99} ] \label{thm:gersten_bowditch}
Let $H$ be a quasiconvex subgroup of a word-hyperbolic group $G$. Suppose $|gHg^{-1} \cap H |$ is finite whenever $g \not \in H$. Then $G$ is relatively hyperbolic with respect to $\Hs$.
\end{theorem}


\begin{figure}[ht]
\begin{center}
\includegraphics[width=0.8\textwidth]{2-manifold-cusp-t.mps}
\end{center}
\caption[Punctured torus group as a relatively hyperbolic group]
{The puncture torus with a complete hyperbolic structure of finite area and the preimage of a cusp neighborhood in the universal cover.}
\label{fig:2-manifold}
\end{figure}

Let $F_2(a, b)$ be .....  (see Figure ~\ref{fig:2-manifold}).


%%%%%%%%%%%%%%%%%%%%%%%%%%%%%%%%%%%%%%%%%%%%%%%%%%%%%%%%%%%%%%%%%%%%%%%%%%%%
%% BIBLIOGRAPHY %%%%%%%%%%%%%%%%%%%%%%%%%%%%%%%%%%%%%%%%%%%%%%%%%%%%%%%%%%%%%%%%%%%%
%%%%%%%%%%%%%%%%%%%%%%%%%%%%%%%%%%%%%%%%%%%%%%%%%%%%%%%%%%%%%%%%%%%%%%%%%%%%

{
\singlespacing

\bibliographystyle{plain}
\bibliography{Xbib}
}


%%%%%%%%%%%%%%%%%%%%%%%%%%%%%%%%%%%%%%%%%%%%%%%%%%%%%%%%%%%%%%%%%%%%%%%%%%%%
%% DEDICATION PAGE  %%%%%%%%%%%%%%%%%%%%%%%%%%%%%%%%%%%%%%%%%%%%%%%%%%%%%%%%%%%%%%%%%%%%%%%%%%
%%%%%%%%%%%%%%%%%%%%%%%%%%%%%%%%%%%%%%%%%%%%%%%%%%%%%%%%%%%%%%%%%%%%%%%%%%%%

%\newpage
\newpage
\pagenumbering{roman}

\thispagestyle{empty}
{\singlespacing
\begin{center}
{
\par
\vspace{1.2in}
%\large
DEDICATION
\par
\vspace{0.17in}
to
\par
\vspace{1.2in}
%%%%%%%%%%%%%%%%%%%%%%%%%%%%%%5
My parents
\par
\vspace{0.17in}
Luis Eduardo Martinez Villarraga, and
\par
\vspace{0.17in}
Esperanza Pedroza Paipa
\par
\vspace{1.2in}
%%%%%%%%%%%%%%%%%%%%%%%%%%%%%%
For
\par
\vspace{0.17in}
Encouraging me to follow my dreams
\par
\vfill
%%%%%%%%%%%%%%%%%%%%%%%%%%%%%
}
\end{center}
}

\end{document}
